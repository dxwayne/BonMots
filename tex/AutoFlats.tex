\documentclass[letter,11pt,oneside]{article}
% /home/wayne/clones/git/ASTR3510_2019/doc/AutoFlats.tex
%%% (occur "\\(\\\\[a-z]*section\\|appendix\\|input\\|\\<include\\>\\)")

%%\documentclass[11pt,twocolumn]{article}
%%\usepackage[inline]{asymptote}   %% Inline asymptote diagrams
%%\usepackage{wglatex}             %% Use this one and kill others.
\usepackage{color}               %% colored letters {\color{red}{{text}}
\usepackage{fancyhdr}            %% headers/footers
%%\usepackage{fancyvrb}            %% headers/footers
\usepackage{datetime}            %% pick up tex date time 
\usepackage{lastpage}            %% support page of ...lastpage
\usepackage{times}               %% native times roman fonts
\usepackage{textcomp}            %% trademark
\usepackage{amssymb,amsmath}     %% greek alphabet
\usepackage{parskip}             %% blank lines between paragraphs, no indent
\usepackage{shortvrb}            %% short verb use for tables
\usepackage{lscape}              %% landscape for tables.
\usepackage{longtable}           %% permit tables to span pages wg-longtable
\usepackage{url}                 %% Make URLs uniform and links in PDFs
\usepackage{enumerate}           %% Allow letters/decorations for enumerations
\usepackage{endnotes}            %% Enhance footnotes/endnotes
\usepackage{listings}            %% Make URLs uniform and links in PDFs
\pdfadjustspacing=1                %% force LaTeX-like character spacing
\usepackage{geometry}            %% allow margins to be relaxed
%%\usepackage{wrapfig}             %% permit wrapping figures.
%%\usepackage{subfigure}              %% images side by side.
\geometry{margin=1in}            %% Allow narrower margins etc.
\usepackage[T1]{fontenc}         %% Better Verbatim Font.
\renewcommand*\ttdefault{txtt}   %% 
%%\usepackage{natbib}   %% bibitems

%% include background image (wg-document-page-background) 

\usepackage{graphicx}            %% Include pictures into a document
%% (wg-texdoc-inserttikz)


\def\documentisdraft{NOTDRAFT}

%% (wg-texdoc-isdraft)
%% (wg-texdoc-insert-fancy-headers)

%%\usepackage[bookmarks]{hyperref} %% Make huperlinks within a PDF
%%\usepackage{makeidx}             %% Make an index uncomment following line
%%\makeindex                       %%.. yeah this one, too. index{key} in text
%%



\definecolor{verbcolor}{rgb}{0.6,0,0}
\definecolor{darkgreen}{rgb}{0,0.4,0}
\newcommand\debate[1]{\textcolor{darkgreen}{DEBATE: #1} \marginpar{\textcolor{red}{DEBATE} }}
\newcommand{\ltodo}[2]{\marginpar{\textcolor{red}{ACTION: #1}\endnote{#2}}}
\renewcommand{\thefigure}{\thesection-\arabic{figure}}
\newcommand{\menu}{\ensuremath{\;\rightarrow\;}}
%%(wg-add-inline-images)  %% add inline images to the mix





%%Begin User Definitions: Hint: ~/.latex.defs and  latex.defs  
%%End User Definitions:
%%(wg-texdoc-adjust-paper-width)
%% (wg-texdoc-insert-hypersetup)



%%%%%%%%%%%%%%%%%%%%%%%%%%%%%%%%%%%%%%%%%%%%%%%%%%%%%%%%%%%%%%%%%%%%%%%%%%%%%


\begin{document}


%% (wg-latex-pretty-title-page)
%% (wg-texdoc-titleblock)

\setcounter{section}{0}
\pagenumbering{arabic}

\ifx\documentisdraft\drafttest
\linenumbers    %%%%%%%%%%%%% DRAFT
\fi



\begin{landscape}
\section*{PyRAF Code} \label{sec:pyrafcode}

See below for a detailed description of the small details and tricks \\
employed. Here is a snippet to cosmic ray correct zero files.

Things should be as simple as they need to be and no simpler!\\
Paraphrase of Einstin with respect to using Bash in PyRAF.

IRAF {\color{verbcolor}{\verb#chpixtype#}} command:

{\color{verbcolor}{\verb#chpixtype myushort.fits myreal.fits real#}}

Works. Converted back to ushort and subtracted from original = 0 everywhere.


\section*{Fix Filenames}

\begingroup \fontsize{10pt}{10pt}
\selectfont
%%\begin{Verbatim} [commandchars=\\\{\}]
\begin{verbatim} 
#
# The bash =~ operator requires the [[ ]] test format and it matches
# patterns that have a space as a sub-string.
#
# shopt -s nullglob; makes the *.*fts test fail without whining about it.

print "Fix the filename and two-dots problem."
!for f in *.fts ; do if [[ "$f" =~ " " ]] ; then  mv "$f" ${f// /_}  ; fi ; done 
!(shopt -s nullglob; for f in *.*.*fts ; do mv  $f  ${f//./_}             ; done)  # change one or more dots to '_'
!(shopt -s nullglob; for f in *-*fts   ; do mv  $f  ${f//-/_}             ; done)  # '-' to '_' ...
!(shopt -s nullglob; for f in *_fts    ; do mv  $f  ${f/%_fts/.fits}      ; done)  # _fts  renamed to .fits
!(shopt -s nullglob; for f in *.fts    ; do mv  $f  ${f//.fts/.fits}      ; done)  # make file a .fits
\end{verbatim}                                                            
\endgroup
%% \end{Verbatim}


\section*{Cosmic Ray Correction.}

\begingroup \fontsize{10pt}{10pt}
\selectfont
%%\begin{Verbatim} [commandchars=\\\{\}]
\begin{verbatim} 
cat l.l
# stop silly questions with crmedian startup.
!echo "unlearn crmedian" > crall.cl  # (erases) and hacks a script called "crall.cl"
!echo "crmedian.crmask    = ''" >> crall.cl
!echo "crmedian.median    = ''" >> crall.cl
!echo "crmedian.residual  = ''" >> crall.cl
!echo "crmedian.sigma     = ''" >> crall.cl
!cat l.l | sed -e 's/\(^.*$\)/crmedian \1 c_\1 /' >> crall.cl

cat crall.cl | less                                 # check and put the script to the log.
cl < crall.cl
\end{verbatim}
\endgroup
%% \end{Verbatim}
\end{landscape}

\clearpage

\begin{landscape}
\begingroup \fontsize{10pt}{10pt}
\selectfont
%%\begin{Verbatim} [commandchars=\\\{\}]
\begin{verbatim} 
#############################################################################
# Make the master SII twilight flat
#############################################################################
# handy boolean expressions:
#
# IMAGETYP "matches" Flat; filter matches SII and binning the same
#    "(IMAGETYP ?= 'Flat' && Filter ?= 'SII' && XBINNING ?= '2')"
#
# Zero files, don't care about binning
#    "(IMAGETYP ?= 'Bias')"
#
# Zero files, we do care about binning!
#    "(IMAGETYP ?= 'Bias' && XBINNING ?= '2')"
#
# Darks: Imagetyp matches Dark and binning is 2 and exposure times same = 1800s
#    "(IMAGETYP ?= 'Dark && XBINNING ?= '2' && EXPTIME = 1800)"
#
# Finally! qualify every file in the list everything hselect 
# Make a handy summary to print for the logbook.
#    hselect *fits DATE-OBS,IMAGETYP,FILTER,EXPTIME,$I yes
#    yes
#
hselect z_c_//@ml.l $I "(Filter ?= 'SII' && IMAGETYP ?= 'Flat')" > l.l   # for zero and cr corrected files...
cat l.l
!rm flat_SII_master.fits
imcombine @l.l flat_SII_master.fits combine=median scale=exposure
#!ds9 flat_SII_master.fits

# hack up our PyRAFic way to normalize a flat using the 100x100 center section.
tmp             = iraf.hselect(Stdout=1,images='flat_SII_master.fits',fields='NAXIS1,NAXIS2',expr=iraf.yes)
(naxis1,naxis2) = map(int,map(int,tmp[0].strip().split()))
section         = '[%d:%d,%d:%d]' % (naxis1//2-50,naxis1//2+50,naxis2//2-50,naxis2//2+50)
mode            = iraf.imstatistics(Stdout=1,images='flat_SII_master.fits'+section,fields='mode',format=iraf.no)

!if test -e n_flat_SII_master.fits; then rm n_flat_SII_master.fits; fi
iraf.imarith(operand1 = 'flat_SII_master.fits',
             op       ='/',
             operand2 = float(mode[0]),
             result   = 'n_flat_SII_master.fits')
!ds9 n_flat_SII_master.fits
\end{verbatim}
\endgroup
%% \end{Verbatim}
\end{landscape}

\clearpage

\begin{landscape}
\section*{Clean up headers}

This is the type of script that the observatory should supply for\\
a telescope/instrument combination. Some software uses .fits \\
sime .fit and some .fts.


\begingroup \fontsize{10pt}{10pt}
\selectfont
%%\begin{Verbatim} [commandchars=\\\{\}]
\begin{verbatim} 
print "Strip wcs from non-science images."
# files *fits > l.l # from all images
# strip Pinpoint WCS from all files... use astrometry.net
hedit @l.l CTYPE1,CRVAL1,CRPIX1,CDELT1,CROTA1               del+ update+ show- ver-
hedit @l.l CTYPE2,CRVAL2,CRPIX2,CDELT2,CROTA2               del+ update+ show- ver-
hedit @l.l CD1_1,CD1_2,CD2_1,CD2_2                          del+ update+ show- ver-
hedit @l.l TR1_0,TR1_1,TR1_2,TR1_3,TR1_4,TR1_5,TR1_6,TR1_7  del+ update+ show- ver-
hedit @l.l TR1_8,TR1_9,TR1_10,TR1_11,TR1_12,TR1_13,TR1_14   del+ update+ show- ver-
hedit @l.l TR2_0,TR2_1,TR2_2,TR2_3,TR2_4,TR2_5,TR2_6,TR2_7  del+ update+ show- ver-
hedit @l.l TR2_8,TR2_9,TR2_10,TR2_11,TR2_12,TR2_13,TR2_14   del+ update+ show- ver-

print "Put science iamge headers into the log."
#hedit *fits OBSERVER 'teamme                               add+ ver- show- update+
hedit *fits OBSERV01 'Grand Fromage'                        add+ ver- show- update+
hedit *fits OBSERV02 'Petite Fromage'                       add+ ver- show- update+
#hedit *fits OBSERV03 'Some Body-Else'                      add+ ver- show- update+

# the IMAGETYP is case sensitive in IRAF
hselect *fits $I "(IMAGETYP == 'Bias Frame')" > l.l
cat l.l
hedit @l.l IMAGETYP zero                                    add+ ver- show- update+

hselect *fits $I "(IMAGETYP ?= 'Dark')" > l.l
cat l.l
hedit @l.l IMAGETYP dark                                    add+ ver- show- update+

hselect *fits $I "(IMAGETYP == 'Flat Field')" > l.l
cat l.l
hedit @l.l IMAGETYP flat                                    add+ ver- show- update+

hselect *fits $I "(IMAGETYP == 'Light Field')" > l.l
cat l.l
hedit @l.l IMAGETYP object                                  add+ ver- show- update+

# Assume SITELAT SITELONG in degrees.
# hselect *fits $I "(SITELAT  ''") > l.l
hedit   *fits OBSGEO-B '(@"SITELAT")'                       add+ ver- show- update+
hedit   *fits OBSGEO-L '(@"SITELONG")'                      add+ ver- show- update+
hedit   *fits OBSGEO-H 366                                  add+ ver- show- update+

# hard way, thanks vendors!
hselect *fits sitelat,sitelong,$I "(SITELAT != '')" 
lattuple = map(float,iraf.hselect(Stdout=1,images='a.fits',
     fields='sitelat',expr=iraf.yes)[0].replace('"','').split())
if(lattuple[0] < 0.0):
   newlat = lattuple[0] - lattuple[1]/60.0 - lattuple[2]/3600.0
else:
   newlat = lattuple[0] + lattuple[1]/60.0 + lattuple[2]/3600.0

iraf.hedit(images='@l.l',fields='OBSGEO-B',value=newlat,
    addonly=iraf.yes,
    verify=iraf.no,
    show=iraf.no,
    update=iraf.yes)

longtuple = map(float,iraf.hselect(Stdout=1,images='a.fits',
   fields='sitelong',expr=iraf.yes)[0].replace('"','').split())

if(longtuple[0] < 0.0):
   newlong = longtuple[0] - longtuple[1]/60.0 - longtuple[2]/3600.0
else:
   newlong = longtuple[0] + longtuple[1]/60.0 + longtuple[2]/3600.0

iraf.hedit(images='@l.l',fields='OBSGEO-L',value=newlong,
    addonly=iraf.yes,
    verify=iraf.no,
    show=iraf.no,
    update=iraf.yes)
# fix altitude
hedit   *fits OBSGEO-H 366                                  add+ ver- show- update+
hselect @l.l  OBSGEO-B,OBSGEO-L,OBSGEO-H,$I yes
\end{verbatim}
\endgroup
%% \end{Verbatim}
\end{landscape}


\appendix
\renewcommand \thesection{\Alph{section}}

\section*{Automatic Cosmic Ray repair and Flat Production}

Some background:

\textbf{\emph{Goal:}} No human interventions -- past setup -- required
for cosmic ray removal (crmedian only works on one file at a time) and
creation of flats (how do divide by mode without interactive action).

Modern cameras with HAD\texttrademark\; and TrueSense \texttrademark\;
technology tend to produce single pixels (Single Pixel Errors or SPEs)
that are essentially ``loud'' noise. The IRAF task
{\color{verbcolor}{\verb#noao.crutil.crmedian#}} will remove SPEs as
well as the occasional cosmic ray -- and may optionally produce
residual files for tracking statistical error-bars.

\textbf{\emph{Action Item:}} The Observatory's engineering staff should make available
a suite of IRAF cl files to fix images from the telescope. For example put
them in a standard location:

{\color{verbcolor}{\verb#/opt/WhatsamattaU/iraf/DasFernrohrZwi/fixum.cl#}}

The steps of observing are 1) Observe 2) Process and 3) Analyze and Publish.

This article looks at the directory structure as starting at a top level
directory called {\color{verbcolor}{\verb#Observations/<date>#}}.
Under this you will have 4 directories: {\color{verbcolor}{\verb#usw#}},
{\color{verbcolor}{\verb#Raw#}}, {\color{verbcolor}{\verb#PreAnalysis#}}
and {\color{verbcolor}{\verb#Analysis#}}. 

The usw directory holds the planning data like DSS images for all
the targets, any UCAC4 ds9 catalog files, other catalog files,
maybe a sub-directory for papers etc. It should hold any ds9
region files, and other configuration files as needed. 

Copy any template {\color{verbcolor}{\verb#cl#}} file to the
{\color{verbcolor}{\verb#usw#}} directory -- and create the very
specific reduce {\color{verbcolor}{\verb#cl#}} (usw/reduce.cl) file(s)
to convert the Raw to the PreAnalysis state. It may take a little
while to get the first reduce.cl file made.

The steps are:

\vspace{-.15cm}
\begin{enumerate}\addtolength{\itemsep}{-0.5\baselineskip}
   \item   convert all the filenames to proper IRAF filenames, no spaces, no
minus signs (use m or p for plus), no long strings of meaningless
zeros and only one dot for the extension.
   \item   Change the keywords, edit out bad header keywords etc.
   \item   Optionally, trim the images for cameras with overscan regions.
   \item   Cosmic ray reduce ALL the files. Might as well do that first.
   \item   Create all masters zeros, darks, flats.
   \item   Apply the masters to the science images.
   \item   WCS solve any images.
   \item   Add the images together for better SNR.
\end{enumerate}

First the Raw images are copied exactly into a permanent location
(Raw). 2) make a copy of these data into a PreAnalysis directory to
process (basic reduction) them to the point of analysis. Copy these
process images into an  Analysis directory -- and use 
reference stars, catalog sources and astrophysics to make publication
ready results.

Once the PreAnalysis directory copy is in place start an empty
cl file located -- copy the
necessary templates for reduction. Start PyRAF and cut and paste
lines from the templates into your reduce.cl script

Flats require zero (bias) subtraction, possibly a scaled
dark correction, and above all cosmic ray correction.
The cosmic ray (CR) correction removes single pixel errors (SPEs).

The master zero requires cosmic ray correction before
combining -- again to address the stuck pixels.

This article contains PyRAF (Python and IRAF) code snippet to create
flat file from twilight flats.

Previously: We cosmic ray corrected ALL the images. Zeros, Flats,
Science Images. 

In the processing steps: 

\vspace{-.15cm}
\begin{enumerate}\addtolength{\itemsep}{-0.5\baselineskip}
   \item   Necessary files are trimmed with a {\color{verbcolor}{\verb#_t#}}
   prepended to the filename.
   \item   All raw images are cleaned with
{\color{verbcolor}{\verb#noao.imred.crutil.crmedian#}} and a
{\color{verbcolor}{\verb#c_#}} prepended to produce filenames as
{\color{verbcolor}{\verb#c_filename.fits#}}
   \item   All master files are created (watch binning)
   \item   All files are then zero subtracted and the filenames have
a {\color{verbcolor}{\verb#z_#}} prepended: {\color{verbcolor}{\verb#z_c_filename.fits#}}
   \item   As processing proceeds the following characters may be used:
\vspace{-.15cm}
\begin{enumerate}\addtolength{\itemsep}{-0.5\baselineskip}
   \item   {\color{verbcolor}{\verb#d_#}} prepended for dark subtraction
   \item   {\color{verbcolor}{\verb#f_#}} prepended for flat division
   \item   {\color{verbcolor}{\verb#n_#}} prepended for normalized flats (usually renamed
{\color{verbcolor}{\verb#flat_band_bin_master.fits#}}.
   \item   All of these intermediate files are cleaned out with:\\
{\color{verbcolor}{\verb#! (export LC_ALL=C; rm [a-z]_*fits 2> /dev/null;)#}}
\end{enumerate}
   \item   Files may be WCS solved as needed.
\end{enumerate}

\textbf{\emph{Example:}} A completely ready file, filename.fits might look like
{\color{verbcolor}{\verb#f_d_c_z_c_t_filename.fits#}}. Reading backwards,
it was trimmed, CR corrected, zero subtracted, dark subtracted, flat
corrected. The summed file is from a collection. With the base file
names in {\color{verbcolor}{\verb#l.l#}}
\begin{quote}
{\color{verbcolor}
\begingroup \fontsize{10pt}{10pt}
\selectfont
%%\begin{Verbatim} [commandchars=\\\{\}]
\begin{verbatim} 
hselect *fits $I "(IMAGETYP ?= 'light')" > l.l
cat l.l
cl << crall.cl  # Force CR correction to all files up front.
imarith c_//@l.l - zeromaster.fits z_c_//@l.l
imarith z_c_//@l.l - darkmaster.fits d_z_c_//@l.l
imarith d_z_c_//@l.l / flat_vband_2x2.fits  f_d_z_c_//@l.l
\end{verbatim}
\endgroup
%% \end{Verbatim}
}
\end{quote}

\textbf{\emph{General rule:}} The {\color{verbcolor}{\verb#l.l#}} file
is used over and over again for a small extent of the processing. It
is a ``list.list'', and should be considered un-trustworthy for any
long period (use {\color{verbcolor}{\verb#cat l.l#}} to peek --- because ``I
always check my work''). Just make a new one if in doubt.

In the above example: IRAF allows ``list of files'' or ``'at' files''. By prepending the
``at cost'' sign ({\color{verbcolor}{\verb#@#}}) to the filename
IRAF will peek inside that file and use each of the files -- one
filename per line, as operands. The {\color{verbcolor}{\verb#@#}}
trick only works with IRAF commands.

You can use bash:

{\color{verbcolor}{\verb#!(cat l.l | xargs ds9 ) &#}}

to read the lines one-at-a-time from l.l and send to a special Unix
{\color{verbcolor}{\verb#xargs#}} command. Xargs makes a long command
line phrase and adds it to rest of its command line before executing.
In this example, ds9 will show all the files in l.l. Hundreds of them
if you have a length l.l file!

IRAF allows you to ``string concatenate'' characters to the beginning
of a file \footnote{end doesn't work so well with extensions in some cases.}
using the {\color{verbcolor}{\verb#//#}} ``concatenation operator''.
The list above has some examples.


\subsection{Using Bash and IRAF to remove cosmic rays and SPEs}

Here we use the Unix output redirection tricks: {\color{verbcolor}{\verb#> l.l#}}
to send output to a new file (will remove old one and make a new one if needed);
then {\color{verbcolor}{\verb#>> l.l#}} to append new output from a task
to the end of this fresh file. This allows several commands to 
``build up'' an new file. 

So, we build a righteous cl script -- here called {\color{verbcolor}{\verb#crall.cl#}}.
Then we use the {\color{verbcolor}{\verb#< crall.cl#}} to input that file
to the IRAF interpreter program {\color{verbcolor}{\verb#cl#}}:

\begingroup \fontsize{10pt}{10pt}
\selectfont
%%\begin{Verbatim} [commandchars=\\\{\}]
\begin{verbatim} 
ls -1 *fits > l.l                    # Get all the files
# stop silly questions with crmedian startup.
!echo "unlearn crmedian" > crall.cl  # (erases) and hacks a script called "crall.cl"
!echo "crmedian.crmask    = ''" >> crall.cl
!echo "crmedian.median    = ''" >> crall.cl
!echo "crmedian.residual  = ''" >> crall.cl
!echo "crmedian.sigma     = ''" >> crall.cl
!cat l.l | sed -e 's/\(^.*$\)/crmedian \1 c_\1 /' >> crall.cl
cat crall.cl                         # I always check my work
cl << crall.cl                       # Force CR correction to all files up front.
\end{verbatim}
\endgroup
%% \end{Verbatim}

The {\color{verbcolor}{\verb#sed -e 's/\(^.*$\)/crmedian \1 c_\1 /'#}} is
a special Unix editor (string ed) to pick up each filename once
and use it twice: once as input file for the crmedian task, and
then with a {\color{verbcolor}{\verb#c_#}} added for the output name!
Crmedian itself does not use the at-files. 


Now:

{\color{verbcolor}{\verb#hselect c_//@l.l $I "(IMAGETYP ?= 'Flat' && FILTER ?= 'alpha' && XBINNING ?= 2)"#}}

shows Flat files binned by 2 for the H$\alpha$ filter -- for all
raw files with cosmic rays fixed.

Here are a few snippets of PyRAF for various steps.




%% use a bibitem approach to the references publications etc.
%% (wg-bibitem)

%%\clearpage
%%\addcontentsline{toc}{section}{References}
%%\renewcommand*{\refname}{My Bibliography and References}
%%\bibliographystyle{plain}	% bibliographystyle{apalike} and \usepackage{natbib}
%%\bibliography{MasterBib}	% expects file "MasterBib.bib"


%%\clearpage
%%\addcontentsline{toc}{section}{Index}
%%\printindex %% www.cs.usask.ca/resources/tutorials/latex/notes/toc-index.pdf

%%\begin{thebibliography}{80}
%%\usepackage{natbib}   %% bibitems
%%\end{thebibliography}

% /home/wayne/clones/git/ASTR3510_2019/doc/AutoFlats.tex

%% (wg-texdoc-endnotes)
\end{document}
