\documentclass[letter,11pt,oneside]{article}

%%% (occur "\\(\\\\[a-z]*section\\|appendix\\|input\\|\\<include\\>\\)")

%%\documentclass[11pt,twocolumn]{article}
%%\usepackage[inline]{asymptote}   %% Inline asymptote diagrams
%%\usepackage{wglatex}             %% Use this one and kill others.
\usepackage{color}               %% colored letters {\color{red}{{text}}
\usepackage{fancyhdr}            %% headers/footers
%%\usepackage{fancyvrb}            %% headers/footers
\usepackage{datetime}            %% pick up tex date time 
\usepackage{lastpage}            %% support page of ...lastpage
\usepackage{times}               %% native times roman fonts
\usepackage{textcomp}            %% trademark
\usepackage{amssymb,amsmath}     %% greek alphabet
\usepackage{parskip}             %% blank lines between paragraphs, no indent
\usepackage{shortvrb}            %% short verb use for tables
\usepackage{lscape}              %% landscape for tables.
\usepackage{longtable}           %% permit tables to span pages wg-longtable
\usepackage{url}                 %% Make URLs uniform and links in PDFs
\usepackage{enumerate}           %% Allow letters/decorations for enumerations
\usepackage{endnotes}            %% Enhance footnotes/endnotes
\usepackage{listings}            %% Make URLs uniform and links in PDFs
\pdfadjustspacing=1                %% force LaTeX-like character spacing
\usepackage{geometry}            %% allow margins to be relaxed
%%\usepackage{wrapfig}             %% permit wrapping figures.
%%\usepackage{subfigure}              %% images side by side.
\geometry{margin=1in}            %% Allow narrower margins etc.
\usepackage[T1]{fontenc}         %% Better Verbatim Font.
\renewcommand*\ttdefault{txtt}   %% 
\usepackage[bookmarks]{hyperref} %% Make huperlinks within a PDF
%%\usepackage{natbib}   %% bibitems

%% include background image (wg-document-page-background) 

\usepackage{graphicx}            %% Include pictures into a document
%% (wg-texdoc-inserttikz)


\def\documentisdraft{NOTDRAFT}

%% (wg-texdoc-isdraft)
%% (wg-texdoc-insert-fancy-headers)

%%\usepackage[bookmarks]{hyperref} %% Make huperlinks within a PDF
%%\usepackage{makeidx}             %% Make an index uncomment following line
%%\makeindex                       %%.. yeah this one, too. index{key} in text
%%



\definecolor{verbcolor}{rgb}{0.6,0,0}
\definecolor{darkgreen}{rgb}{0,0.4,0}
\newcommand\debate[1]{\textcolor{darkgreen}{DEBATE: #1} \marginpar{\textcolor{red}{DEBATE} }}
\newcommand{\ltodo}[2]{\marginpar{\textcolor{red}{ACTION: #1}\endnote{#2}}}
\renewcommand{\thefigure}{\thesection-\arabic{figure}}
\newcommand{\menu}{\ensuremath{\;\rightarrow\;}}
\newcommand{\dhl}[1]{{\color{verbcolor}{\texttt#1}}}
\definecolor{wglightgreen}{rgb}{0.88, 0.58, 0.88}
\newcommand{\wgtextbox}[1]{\noindent\fcolorbox{darkgreen}{wglightgreen}{%
    \minipage[t]{\dimexpr0.80\linewidth-2\fboxsep-2\fboxrule\relax}
        {#1}
    \endminipage}}
%%(wg-add-inline-images)  %% add inline images to the mix





%%Begin User Definitions: Hint: ~/.latex.defs and  latex.defs  
%%End User Definitions:
%%(wg-texdoc-adjust-paper-width)
%% (wg-texdoc-insert-hypersetup)



%%%%%%%%%%%%%%%%%%%%%%%%%%%%%%%%%%%%%%%%%%%%%%%%%%%%%%%%%%%%%%%%%%%%%%%%%%%%%


\begin{document}


%% (wg-latex-pretty-title-page)
%% (wg-texdoc-titleblock)

\setcounter{section}{0}
\pagenumbering{arabic}

\ifx\documentisdraft\drafttest
\linenumbers    %%%%%%%%%%%%% DRAFT
\fi

\section{Planning Observations}

One must first choose the target and reference stars. Some targets
set their own time of observation. The periodic variable stars,
asteroid occultations, some transient events catch your attention
but the season places the target close to the Sun or Moon and
a handful of other inconvient truths!

The critical aspects for a target include altitude (airmass), light
pollution (car dealerships, the Moon, planets, trees etc). These are
aaspects you have to determine for yourself. Once a few targets
are selected, developing the tactics to observe them is straight forward.


The ideal target list starts about 1 hour to the west of the meridian
and you work east for the night. Too far east puts the target into


In this article we will look at using the Smithsonian Astrophysical
Observatory tool SAOImage commonly referred to by its nickname
ds9.SAOImage/ds9 is ``platform agnostic'' meaning it runs on Windows,
Macs, Linux -- even the Raspherry Pi. Written in TCL/Tk with some
C/C++ for heavy lifting under the hood, ti is the latest of a long
line of image examination programs that originated with IRAF and is
still used as the interactive image gateway for IRAF analysis
programs.  It uses both XPA and SAMP for inter-process
communications. It will tie in with TOPCAT, Aladin, GLUE, Python 2.x
and 3.x as well as a few other programs for a complete and extremely
powerful work environment.

But, for simple planning it can't be beat.

This is the goal. Observe V694 Mon, photometry and spectroscopy.
If the name is known to SIMBAD you do not need to know the
RA/DEC per se for ds9.


Start ds9.

Analysis; Image Servers; DSS ESO
Up pops a little dialog box.

Under it's menus:
Survey; DSS2-Red
Set the Width and Height to 45x45 arcminutes
Enter V694 Mon into the Object field
and hit the Retrieve button

An image will fill into the main ``frame''

Then under
Analysis; Catalogs; Database;
Select SIMBAD

Up pops a ``Catalog Tool'' window, and up to
5000 entries from SIMBAD will be loaded.

The target shows up with its MAIN\_ID set to \dhl{EM* MWC 560}.

\subsection{Using the Catalog Tool}

Great the target has an odd name, which of the 141\footnote{Your count
  may vary} targets returned is the one we want?

In the ``Catalog Tool'' window, middle frame, you see \dhl{Sort}.
The little box is blank, so click on that and select \dhl{OTYPE\_S}
and the \dhl{Increase} radio button. Under \dhl{OTYPE\_S} you
see lots of stars, and down at the bottom the \dhl{Symbotic*}.
Note the name is \dhl{EM* MWC 560}.

Now some fancy dancing. Click on the line and the
circle for that star over in the image will highlight!

\subsection{Fields of View}

Use Edit; Region from the main menu, then Region; Shape;
Box. Somewhere in the main image, say around \dhl{EM* MWC 560}
just right-click the mouse. A box will appear. Double click
inside that box and its dialog pops up. Set the size to
your guide star's size. Mine is 6x6 arcminutes. I set the
Units selection box to the side of the \dhl{Size} to ``arcmin''.

Now with this as the only region, save it. You can use it forever.

Region; Save Regions; and then answer where you want to put it.

\subsection{Some High-level Planning Ideas}

I keep a high-level planning directory at \dhl{~/iraf/usw}. IRAF is
not needed, but since I use it heavily that is where I choose to
keep my high-level ``stuff''.

I keep a directory \dhl{~/Observations}, and under that I have
past years, and all the nights from the current year.

I name my directories ddMMMyyyy where dd is the one or two digit
day of the month, MMM is the three letter designation for the Month,
and yyyy is the 4 digit (lets not do Y2K again please) year. The
date is ``sunset'' at the observatory where the data are taken.


It makes
autocomplete easier and it is very easy to read when I'm half sentient.

Under that I create directories: Focus, RawData and usw\footnote{Many
  operating systems over-load etc (et cetra) for system purposes. I use
  the German equivalent ``usw'' ``und so weider'' for and fo forth instead.
  This means I realize what is going on with something that is seldom if ever
  seen.}. Into the local ``usw'' I create or copy things I want for that
night. So from \dhl{~/iraf/usw} I copy the region files and other data
that relates to the instrument package(s) that I use for that night.

Into this directory I save DSS images, jpeg and other images, catalog
data, notes, and anything that I will need at the scope.






%%\appendix
%%\renewcommand \thesection{\Alph{section}}

%% use a bibitem approach to the references publications etc.
%% (wg-bibitem)

%%\clearpage
%%\addcontentsline{toc}{section}{References}
%%\renewcommand*{\refname}{My Bibliography and References}
%%\bibliographystyle{plain}	% bibliographystyle{apalike} and \usepackage{natbib}
%%\bibliography{MasterBib}	% expects file "MasterBib.bib"


%%\clearpage
%%\addcontentsline{toc}{section}{Index}
%%\printindex %% www.cs.usask.ca/resources/tutorials/latex/notes/toc-index.pdf

%%\begin{thebibliography}{80}
%%\usepackage{natbib}   %% bibitems
%%\end{thebibliography}

% /home/wayne/iraf/smtsci/doc/Planning.tex

%% (wg-texdoc-endnotes)
\end{document}
