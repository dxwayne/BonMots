\section{Spectrum Flat Fields}

Consider a slit of width $W$, and length $L$ being illuminated by an
internal collection of lamps with intensity $I$ proximal to the
slit\footnote{The lamps will move with the spectrograph and will not
  turely represent the beam from the OTA.}. Take n number of positions
$p$ along the slit.


Hi Frank,

Greetings from India! 

Our new 2 meter telescope has started and we have been trying
observations with its Imaging Spectrograph to observe many of our
CFLIB stars. While reducing the spectroscopic data for the red part
(one grism has range 600-850nm) and we see a lot of fringes (since its
a thin CCD). Can you suggest an IRAF task which can effectively remove
the fringes -- (you had done it for our CFLIB spectra)?

I have attached a .ms.fites file of HR5501 IRAF std star which shows
the effect of fringes in the red side.

Looking forward to your reply,

Regs Ranjan


The usual steps for handling fringing is to remove it during flat
fielding.  So the flat field is first processed by CCDPROC and then a
"response" flat is created using the task RESPONSE.  The idea in this
task is to fit a smooth function of wavelength for the average flat
(automatically generated from the 2D flat) and then divide each pixel
in the 2D flat with the appropriate function value.  The key is to use
a fitting order which is high enough to remove the general shape of
the flat field lamp but not so high as to fit the fringing.  By
dividing by the low order fit the general shape of the flat field will
not be applied to the science exposures but the higher frequency
fringing will be divided out of the science data.  The response flat
you generate is then applied using CCDPROC with the response flat
specified as the flat instead of the original flat.

Note that if you are going to do flux calibration then it really
doesn't matter if the shape of the flat field spectrum is introduced
to the science data since the flux calibration step will handle that.

So in that case you can either just skip RESPONSE and directly divide
by the flat field (in CCDPROC) or use a very low order function like a
constant to just normalize the flat field.

There is no other regular way to handle the fringing other than
dividing by a flat field.  It would be possible to divide by the
extracted flat field.  In other words you could do the flat fielding
and fringe removal in 1D instead of in 2D.

Cheers,
Frank
