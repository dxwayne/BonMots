\section{Using ds9 and a real database}

ds9 will attach to mysql. There is legacy code here from an old project
that does just that. However, MySQL\texttrademark requires a lot of
system-dependent administration and imposes undue compute burdens.

The easier path is to create a SQLite3 database. This is can be done
in memory (real fast and very temporary) or it can be its own format
file. This file may be 'dumped', munged, and imported elsewhere.

The trick is to attach the SQLite3 tcl package. This is tricky.
This example is within the ds9's anaconda virtual python package.

\begingroup \fontsize{10pt}{10pt}
\selectfont
%%\begin{Verbatim} [commandchars=\\\{\}]
\begin{verbatim} 
lappend ::auto_path {/home/wayne/anaconda3/envs/iraf27/lib}
package require sqlite3
\end{verbatim}
\endgroup
%% \end{Verbatim}

\textbf{\emph{Prose:}} Hey, ds9 you sourced my file that adds the path
to SQLite3 within the anaconda \dhl{iraf27} virtual python
environment. The next 'package' statement loads \dhl{sqlite3} into the
mix.

Now we are free to use all the power that database offers without
the administration stuff.

Conventions are in order.

For each database we create, we prepend a \dhl{ds9db_} to the file
name to let the system track them down later. We like to leave
lots of databases, one each for the area where ds9 is being used.
This lets you archive and share just the stuff related to the
observations/project you have.

The \dhl{sasiraf/ds9} has a sqlite3.ini file with a collection of
pre-made handy functions -- like \dhl{jpldate} that returns ddMonYYYY.
Hack this to taste.
