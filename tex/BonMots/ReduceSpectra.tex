% BonMots/ReduceSpectra.tex
\section{Reduce Spectral Sequence}

A sequence consists of three images, a comparison lamp, a flat and a science target star.
This section does not deal with reduction of a extended source across a long slit.

The files, (a10234) is a serial number added to as a prefix to the images
taken, in order specified by the DATE-OBS keyword):

   \dhl{a10234\_Comp.fits} \\
   \dhl{a10235\_Flat.fits} \\
   \dhl{a10236\_Vega.fits}

The very first thing to do is to use \dhl{epar apall} to check the
apall parameters. We aim to produce a spectrum with 4 bands:
\vspace{-.15cm}
\begin{enumerate}\addtolength{\itemsep}{-0.5\baselineskip}
%\setcounter{enumi}{N}
   \item  The reduced spectrum
   \item  The extracted spectrum sans any corrections
   \item  The sky background information
   \item  The variance for the spectrum.
\end{enumerate}

Parameters are shown and discussed in section \ref{sec:Apalllpars} and in
\ref{fig:ApallParameters}.


First extract the star's trace from the a10234\_Vega.fits file. This creates
a file in the database directory called:

\dhl{apa10234\_Vega}

with contents shown in figure \ref{fig:ApextractStar}.

Use the command:

\dhl{apall a10236\_Vega.fits} 

Be prepared to do several manual steps, and to answer a few questions.


Find apertures for Vega_Light_003? ('yes'): \llbox{CR} \\
Number of apertures to be found automatically (1): \llbox{CR} \\
Resize apertures for Vega_Light_003? ('yes'): \llbox{CR} \\
Edit apertures for Vega_Light_003? ('yes'): \llbox{CR} \\

Window Pops Up, showing the sum across the rows of all the data.
This gives a clear indiciation of where the trace's aperture
is to be defined. Use:

- \llbox{m} to mark an aperture
- \llbox{d} to delete an aperture
- \llbox{u} for the upper and \llbox{l} for the lower sides of the aperture.
- \llbox{w}-\llbox{e}-\llbox{e} to zoom into the image to get a clear idea
   of where to set the extents for the aperture.
- \llbox{w}-\llbox{a} to zoom all the way back out.

\subsection{Fit the Aperture}
-\llbox{t} - trial trace
-\llbox{s}-\llbox{s}  set segment limits on trace
-\llbox{f} do the fit, check residuals
-\llbox{d} remove few bad points
-\llbox{w}-\llbox{e}-\llbox{e} and \llbox{w}-\llbox{a} allow zooming in to
find bad points
-\llbox{f}-\llbox{q} do the final fit and move on to next step.


\textbf{\emph{Note:}} The questions are often asked in the command window, with the
answer accepted within the GUI/Display for the apall task.


The next trick is to use the trace math in the \dhl{apa10234\_Vega} and
extract the corresponding pixels from the comp lamp and the flat lamp.

This is done with 


\subsection{Background Fit}

The background is the area just above and just below the trace
where the sky background is extracted. With spectrographs with
extreme smile (curved nature of the spectral lines, evicenced
by the calibration image, is pronounced) blurs the sky contamination.
Keep the area about the same as the combined distance between u-l
used to set the trace's width. 

\llbox{z}-\llbox{z} - remove the two segments \\
\llbox{s}-\llbox{s} - set the left segment \\
\llbox{s}-\llbox{s} - set the right segment \\
\llbox{f} - fit for background \\

\llbox{h}-\llbox{l}	Graph keys.  Defaults are h=(x,y), i=(y,x), j=(x,r), k=(x,d), l=(x,n)

\llbox{q}



\subsection{Trace aperture}

\dhl{Trace apertures for Vega_Light_003?} \\
\dhl{Fit traced positions for Vega_Light_003 interactively?} \\
\dhl{Fit curve to aperture 1 of Vega_Light_003 interactively} \\

\vspace{-.15cm}
\begin{enumerate}\addtolength{\itemsep}{-0.5\baselineskip}
%\setcounter{enumi}{N}
   \item  \llbox{s}-\llbox{s} - sets segment to define fit extents
   \item  \llbox{d} - remove any errant points
   \item  \llbox{f}   fit the aperture
   \item  \llbox{q} move on to the next step
\end{enumerate}


\subsection{Update the Database}

Little to do here, except let IRAF do its thing. Questions appear
in the command window, answers go into the text area on bottom
of graphics area.


\dhl{Write apertures for Vega_Light_003 to database} \\
\dhl{Extract aperture spectra for Vega_Light_003?} \\
\dhl{Review extracted spectra from Vega_Light_003?} \\
\dhl{Review extracted spectrum for aperture 1 from Vega_Light_003?}

A preview (not splot) shows the result.

\llbox{q} finish, move to next job.

Good idea to \dhl{splot} the result, and double check residuals.


\figure{begin}
\begingroup \fontsize{10pt}{10pt}
\selectfont
%%\begin{Verbatim} [commandchars=\\\{\}]
{\color{green}
\begin{verbatim} 
# Thu 22:38:41 11-May-2023
begin   aperture Vega_Light_006 1 3300. 2034.009
        image   Vega_Light_006
        aperture        1
        beam    1
        center  3300. 2034.009
        low     -3299. -4.322676
        high    2952. 5.233746
        background
                xmin -78.09895
                xmax 68.85063
                function chebyshev
                order 13
                sample  -78.09895:-64.55042 51.1333:68.85063
                naverage 1
                niterate 3
                low_reject 3.
                high_reject 3.
                grow 0.
        axis    2
        curve   17
                1.
                13.
                1490.
                5330.
                4.210632
                -3.841412
                5.277158
                -3.067528
                0.8136517
                -1.038884
                -0.2787266
                -0.003336905
                0.06488369
                0.1270491
                0.04203865
                0.04923941
                0.008242447
\end{verbatim}
}
\endgroup
%% \end{Verbatim}
\caption{Contents of a star's aperture extraction.}
\label{fig:ApextractStar}
}




















%
