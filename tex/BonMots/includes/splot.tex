\subsection{Splot Help}

Handy tasks: 

\begin{table}[h!]
%\phantomsection
%\addcontentsline{toc}{section}{ TOC CAPTION}
% \setlength{\belowcaptionskip}{6pt} % adjust space under caption abovecaptionskip
% \renewcommand{\arraystretch}{1.3} % adjust line spacing
%\small{
%\begin{minipage}{\textwidth}     % for footnotes in table.
%\caption[TOC]{Splot Key Sequences}
\centering
\begin{tabular}{| l | l |l |}
%\MakeShortVerb{\|}
%\multicolumn{n}{fmt}{text for merged cols}
\hline
Action  & Key Sequence & Effect  \\
\hline
   fit continuum:  & \llbox{t-fq}                           & continuum fit        \\ 
   write NEW file  & \llbox{i} \dhl{cont\_<newfilename>}    & 'write' new contunuum file \\
\hline
   window:         & cursor at left edge\llbox{a} cursor at right edge \llbox{a} & zoom window  \\ 
\hline
   title:          & \dhl{:/sysid no} \crreturn             & ~ ~ remove existing  \\ 
                   & \dhl{:/title <txt>} \crreturn          & ~ ~ line 1 title     \\ 
                   & \dhl{:/subtitle <subtxt>} \crreturn    & ~ ~ line 2  subtitle \\ 
                   & \dhl{:/comment <commenttxt>} \crreturn & ~ ~ line 3 comment   \\ 
                   & \llbox{r}                              & SHOW NEW TITLE       \\ 
\hline
  see values:      & \dhl{spacebar} & show x,values... \\
%% ones-based: \cline{a-b}
\hline
%%\DeleteShortVerb{|}
\end{tabular}
%%\end{minipage}    %% for footnotes  r@{.}l 
\caption{Splot Key Sequences}
\label{table:SplotKeySequences}
%%} % end small etc
\end{table}

\begingroup \fontsize{10pt}{10pt}
\selectfont
%%\begin{Verbatim} [commandchars=\\\{\}]
\begin{verbatim} 
? - This display                          r - Redraw the current window
/ - Cycle thru short help on stat line    s - Smooth (boxcar)
a - Autoexpand between cursors            t - Fit continuum(*)
b - Toggle base plot level to 0.0         u - Adjust coordinate scale(*)
c - Clear and redraw full spectrum        v - Velocity scale (toggle)
d - Deblend lines using profile models    w - Window the graph
e - Equiv. width, integ flux, center      x - Connects 2 cursor positions
f - Arithmetic functions: log, sqrt...    y - Plot std star flux from calib file
g - Get new image and plot                z - Expand x range by factor of 2
h - Equivalent widths(*)                  ) - Go to next spectrum in image
i - Write current image as new image      ( - Go to previous spectrum in image
j - Fudge a point to Y-cursor value       # - Select new line/aperture
k - Profile fit to single line(*)         % - Select new band
l - Convert to F-lambda                   $ - Toggle wavelength/pixel scale
m - Mean, RMS, snr in marked region       - - Subtract deblended fit
n - Convert to F-nu                       , - Down slide spectrum
o - Toggle overplot of following plot     . - Up slide spectrum
p - Convert to wavelength scale           I - Interrupt task immediately
q - Quit and exit                   <space> - Cursor position and flux

(*) For 'h' key: Measure equivalent widths
    a - Left side for width at 1/2 flux   l - Left side for continuum = 1
    b - Right side for width at 1/2 flux  r - Right side for continuum = 1
    c - Both sides for width at 1/2 flux  k - Both sides for continuum = 1

(*) For 'k' key: Second key may be used to select profile type
    g - Gaussian, l - Lorentzian, v - Voigt, all others - Gaussiank

(*) For 't' key: Fit the continuum with ICFIT and apply to spectrum
    / = normalize by the continuum fit
    - = subtract the continuum fit (residuals)
    f = replace spectrum by the continuum fit
    c = clean spectrum of rejected points
    n = do the fitting but leave the spectrum unchanged
    q = quit without fitting or modifying spectrum

(*) For 'u' key: Adjust the coordinate scale by marking features
    d = apply doppler correction to bring marked feature to specified coord.
    l = set linear (in wavelength) coordinates based on two marked features
    z = apply zero point shift to bring marked feature to
	specified coordinate

The colon commands do not allow abbreviations.

:# <comment>      - Add comment to log file
:dispaxis <val>   - Change summing parameter for 2D images
:log	          - Enable logging to save_file
:nolog            - Disable logging to save_file 
:nsum <val>       - Change summing parameter for 2D images
:show	          - Show full output of deblending and equiv. width measurments
:units <value>	  - Change coordinate units (see below)

:label  <label> <format> - Add label at cursor position
:mabove <label> <format> - Add tick mark and label above spectrum
:mbelow <label> <format> - Add tick mark and label below spectrum
    The label must be quoted if it contains blanks.  A label beginning
    with % (i.e. %.2f) is treated as a format for the x cursor position.
    The optional format is a gtext string (see help on "cursors").
    The labels are not remembered between redraws.

:auto [yes|no]    - Enable/disable autodraw option
:zero [yes|no]    - Enable/disable zero baseline option
:xydraw [yes|no]  - Enable/disable xydraw option
:hist [yes|no]    - Enable/disable histogram line type option
:nosysid [yes|no] - Enable/disable system ID option
:wreset [yes|no]  - Enable/disable window reset for new spectra option
:flip [yes|no]    - Enable/disable dispersion coordinate flip
:overplot [yes|no]- Enable/disable permanent overplot mode

:/help  Get help on GTOOLS options
:.help	Get help on cursor mode options


				UNITS

The units are specified by strings having a unit type from the list below
along with the possible preceding modifiers, "inverse", to select the
inverse of the unit and "log" to select logarithmic units. For example "log
angstroms" to plot the logarithm of wavelength in Angstroms and "inv
microns" to plot inverse microns.  The various identifiers may be
abbreviated as words but the syntax is not sophisticated enough to
recognized standard scientific abbreviations except as noted below.

	   angstroms - Wavelength in Angstroms
	  nanometers - Wavelength in nanometers
	millimicrons - Wavelength in millimicrons
	     microns - Wavelength in microns
	 millimeters - Wavelength in millimeters
	  centimeter - Wavelength in centimeters
	      meters - Wavelength in meters
	       hertz - Frequency in hertz (cycles per second)
	   kilohertz - Frequency in kilohertz
	   megahertz - Frequency in megahertz
	    gigahertz - Frequency in gigahertz
	         m/s - Velocity in meters per second
	        km/s - Velocity in kilometers per second
	          ev - Energy in electron volts
	         kev - Energy in kilo electron volts
	         mev - Energy in mega electron volts
		   z - Redshift

	          nm - Wavelength in nanometers
	          mm - Wavelength in millimeters
	          cm - Wavelength in centimeters
	           m - Wavelength in meters
	          Hz - Frequency in hertz (cycles per second)
	         KHz - Frequency in kilohertz
	         MHz - Frequency in megahertz
	         GHz - Frequency in gigahertz
		  wn - Wave number (inverse centimeters)

The velocity and redshift units require a trailing value and unit defining the
velocity zero point.  For example to plot velocity relative to
a wavelength of 1 micron the unit string would be:

	km/s 1 micron
\end{verbatim}
\endgroup
%% \end{Verbatim}
