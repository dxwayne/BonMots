\section{Composit Flats}

Composit flats made by the Kzin ring, use multiple illumination
sources in combination to provide sufficient counts in the blue end --
where sensors are insensitive and the blackbody response of bulbs are
also low.  When working with a ``composite'' flats, the goal is to
bring the edges of the slit \dhl{at each wavelength} up to the level
of where the science trace will be.

Look at a calibration lamp image, or a sky image with strong skylines.
You will note that there is tilt, smile and a higher-order spacing
from the blue end to the red end caused by the changes in wavelength
convolved with the physics of the dispersing optic and optical
projection within the instrument. In other words, wavelength is
not constant along the trace, and this relationship is unique to
each column (row for DISPAXIS=2). Therefore generating a master
flat is a nightmare!

Flats require matching calibration lamp images. The OBJECT should
read "flatcal". The RA/DEC fields should agree between the flat
and cal images (indicator that the same flexure is in effect). Exposures
should be very close in time (indicator that temperature differentials
are not present). Each cal sequence should include a matching flat
exposure).


Here is a PyRAF snippet for the work:

\begingroup \fontsize{10pt}{10pt}
\selectfont
%%\begin{Verbatim} [commandchars=\\\{\}]
\begin{verbatim} 
chdir OBS$/Flats
ll a*fits
imarith  @l.l - OBSROOT$/usw/zmaster.fits z_//@l.l  # zero sub the darks
!rm -f flatmaster.fits fmaster.fits
flatcombine z_@l.l output=flatmaster.fits statsec="[*,404:422]" combine=average ...
   ... reject=crreject scale=mode
f = fits.open('flatmaster.fits')
d = f[0].data.T   # back into IRAF coordinates
h = f[0].header
tracemean =  d[:,404:422].mean(axis=1)
our_figure()                                      # clear the plot
our_plot(tracemean,np.arange(*tracemean.shape))   # reverse back to numpy coords
our_plot_show()                                   # finished overplotting(none), render.
fdata = d / tracemean[:,None]
if('OBJECT' in h):
   h['OBJECT'] = ('FMASTER')
newfits('fmaster.fits',fdata.T,h)
!cp fmaster.fits $OBSROOT/usw/fmaster.fits
\end{verbatim}
\endgroup
%% \end{Verbatim}

The functions \verb=our_figure(),our_plot(),our_plot_show(),= and \verb=newfits()=
are included in the Pyraflogin.pyraf confguration script.
\index{pyraflogin!our\_figure()} \index{pyraflogin!our\_plot()} \index{pyraflogin!our\_plot\_show()} \index{pyraflogin!newfits()}. The \verb=newfits()= function creates the \dhl{fmaster.fits} file.
plot functions are used for checking results only. The function \dhl{ll} is simply
a foreign function \verb=ls -1 > l.l && cat l.l=, usually added to \dhl{loginuser.cl}
initialization script.

The assumption is that smile and tilt are not that severe, and that the entire column
may be treated as the same wavelength.
