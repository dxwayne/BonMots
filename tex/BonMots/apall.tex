\subsection{IRAF apall task}  \label{sec:apall}

\example For a \dhl{comparison}, \dhl{flat} and \dhl{science} image set you
want to develop the database using the trace(s) of stars/regions from
the science image. Then apply that trace to the comparison and flat
images.

The apall task requires a few things:
\vspace{-.15cm}
\begin{enumerate}\addtolength{\itemsep}{-0.5\baselineskip}
%\setcounter{enumi}{N}
   \item   hedit *fits DISPAXIS 1 add+ update+ ver- show-
   \item   trim the files to tight red/blue fits. 
   \item   given the pixel size:
\vspace{-.15cm}
\begin{enumerate}\addtolength{\itemsep}{-0.5\baselineskip}
%\setcounter{enumi}{N}
   \item   lower and upper to fit seeing size of the trace. This may be changed
   using interactive=yes.
   \item   line needs to be the physical (or image) location of a bright piece of
   the trace. This starts the process of finding the trace.
   \item   backgrouind b\_sample = "-20:-15,15,20" sets the little background
   ``segments'' \llbox{s}-\llbox{s} key sequence
\end{enumerate}
   \item   format = "onedspec"
   \item   extras = yes -- this provides 4 bands with quality information for
           extracted spectra.
   \item   function chebyshev (avoid Runge phenomena)
   \item   order 5 to, say, 15
   \item   background = "average"
   \item   weights = "variance"
\end{enumerate}


and perhaps a few special to your needs.

\subsection{Aperture Marking}

You can use \llbox{w}-\llbox{e}-\llbox{e} to set the extent of a ``window''.
The first \llbox{e} at the lower left and the second \llbox{e} at the upper right.

The \llbox{l} and \llbox{u} set the lower/upper (left/right) edges of
the area to sample.  Set this according to the wide part of the trace
near the bottom.

Then use \llbox{b} to enter background area definition. Use \llbox{z} \llbox{z}
to clear out the lower and upper regions. Then set the first area by putting
the cross-cursor at the bluest side and \llbox{s} move over a small distance
and \llbox{s} again -- this sets the first region. The region should be close
to the spectrum to avoid cross-contamination because of tilt and smile. Repeat
on the red side of the spectrum. Next use \llbox{f} to fit the results. A residuals
will be reported in the banner, and a curve is drawn. The flatter the curve the better.
Now \llbox{q} to quit background fitting.

\llbox{q} to end the aperture fitting.

\dhl{Trace apertures for Vega\_Light\_006?} is the next question. yes \\
\dhl{Fit traced positions for Vega\_Light\_006 interactively?} yes \\
\dhl{Fit curve to aperture 1 of Vega\_Light\_006 interactively} yes \\

\textbf{\emph{Trick}} Use the  \llbox{s}- \llbox{s} trick to set extents 
to where the trace is well represented (brighter and not noisy).

Use the \llbox{k} key to change the display. Then \llbox{w}-\llbox{e}-\llbox{e}
to zoom into the straight part; and \llbox{d} to delete errant points. (hot
pixels or cosmic rays). Then \llbox{f} to do another fit -- check the residuals
in the banner. Repeat until your are satisfied, then \llbox{q} to end the fit
curve portion.

\dhl{Write apertures for Vega\_Light\_006 to database}  yes\\
\dhl{Extract aperture spectra for Vega\_Light\_006?} yes\\
\dhl{Review extracted spectra from Vega\_Light\_006?} yes

The spectrum is plotted, but this is not the splot task. If it
looks decent, then jump into splot and really take a close look
to make sure the extractions is correct.
