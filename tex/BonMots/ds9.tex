\section{ds9}

The SAOImage/ds9 package is a powerful and extensible tool for image
viewing.  It integrates well with a host of other programs via the
SAMP or XPA inter-process communications facilities. The programs
include PyRAF/IRAF, TOPCAT, Aladin, Glue and various levels of Python
programs.

The basic usage is covered elsewhere. Here we explore a suite of extension
examples offered via GitHUB\footnote{\url{https://github.com/dxwayne/sasiraf}}.
SAOImage makes one critical mistake on *inx systems -- it uses 'hidden'
files -- a trick best not used even by experts except when necessary.
The key files:

\begingroup \fontsize{10pt}{10pt}
\selectfont
%%\begin{Verbatim} [commandchars=\\\{\}]
\begin{verbatim} 
$HOME/.ds9         - special directory with ds9 related code
$HOME/.ds9.X.y.prf - file carrying Edit -> Preferences values.
$HOME/.ds9.ini     - the default file for initialization code.
./.ds9.ini         - a local directory ds9.ini file.
\end{verbatim}
\endgroup
%% \end{Verbatim}

\subsection{Extending ds9}

Tcl/TK is the ``Tool command language'' and the ``toolkit'' for gui and
other shared operations. The basic details are left to the Internet.
The trick here:

\begingroup \fontsize{10pt}{10pt}
\selectfont
%%\begin{Verbatim} [commandchars=\\\{\}]
\begin{verbatim} 
Download the source
grep -Hins 'something' *tcl - return filename:lineno, stuff case insensitive
See how ds9 does it
Make your own function or override one already in use.
\end{verbatim}
\endgroup
%% \end{Verbatim}

The main problem is the ds9 maintainers are using too many 'cute'
bits of the language -- namespaces, etc and adding too many knobs
to make hacking it easy.

\subsubsection{Basic Hacks}

\textbf{\emph{Prose:}} Hey, ds9, when you wakeup from a GUI
launch I want to get to my data fast -- and not wade through
the moras of hidden system files. I'm making my own fake
system file (BadThing\texttrademark) to get me there fast
with a 'soft' link. \index{ds9!link/shortcut for data}.

\begingroup \fontsize{10pt}{10pt}
\selectfont
%%\begin{Verbatim} [commandchars=\\\{\}]
\begin{verbatim} 
ln -s /Observations/4Jul2019/PreAnalysis .AAAAToday
\end{verbatim}
\endgroup
%% \end{Verbatim}


Don't add your hack code directly to \dhl{ds9.ini} add it to a
side file and then include that file in \dhl{ds9.ini} like
\index{ds9!tcl!ds9.ini hack} this:


\begingroup \fontsize{10pt}{10pt}
\selectfont
%%\begin{Verbatim} [commandchars=\\\{\}]
\begin{verbatim} 
set ::wgdebug 0
# user specific
#source "$::env(HOME)/wgds9/lco.ds9.ini"
#source "$::env(HOME)/wgds9/notes.ds9.ini"
source "$::env(HOME)/wgds9/titan.ds9.ini"
#source "$::env(HOME)/wgds9/ds9jfind.ini"
source "$::env(HOME)/wgds9/hh212.ds9.ini"
\end{verbatim}
\endgroup
%% \end{Verbatim}

\textbf{\emph{Prose:}} Hey, make a special global variable called
\dhl{wgdebug} to help me out later. The \dhl{\#} character is a
commment -- drop all characters to the end of current line. My main
machine is called \dhl{titan}. My special \dhl{location} for my hacks
is \dhl{~/wgds9} (non-hidden) directory. When, not if, something steps
on .ds9.ini then I don't lose my tricks. \textbf{\emph{Hint:}} create
a special github account for yourself and keep this important info in
the cloud.

\subsubsection{Example Hacks}

\vspace{-.15cm}
\begin{enumerate}\addtolength{\itemsep}{-0.5\baselineskip}
   \item   Extend Help. Complicated -- have to hack into menu system.

   \item   JFind -- a Tcl/TK gui program that makes finding files on Unix easier.
Architected so its main guts can be added to ds9 directly!

   \item   Dialogs

\vspace{-.15cm}
\begin{enumerate}\addtolength{\itemsep}{-0.5\baselineskip}
   \item   Simple: List the files in current frames
   \item   Hard: tie to database for quality descriptions
\end{enumerate}



   \item   Tie to a database
\vspace{-.15cm}
\begin{enumerate}\addtolength{\itemsep}{-0.5\baselineskip}
   \item   Easy way -- write SQL insert code!
   \item   Hard way -- start a connection to the engine.
\end{enumerate}

\end{enumerate}



\subsection{Database tie-in}

\dhl{sudo apt-get install libsqlite3-tcl}
\dhl{sudo apt-get install tcl tcl-dev }


MySQL is not a very good database. The better one is PostgreSQL.
However, Tcl does tie to the MySQL engine rather well.


See the path's for ds9:

\begingroup \fontsize{10pt}{10pt}
\selectfont
%%\begin{Verbatim} [commandchars=\\\{\}]
\begin{verbatim} 
join $::auto_path
\end{verbatim}
\endgroup
%% \end{Verbatim}

\begingroup \fontsize{10pt}{10pt}
\selectfont
%%\begin{Verbatim} [commandchars=\\\{\}]
\begin{verbatim} 
/zvfsmntpt/tcl8.6
./zvfsmntpt
/home/wayne/lib
/tmp/ubuntu18/SAOImageDS9/lib
./zvfsmntpt/tk8.6
./zvfsmntpt/tk8.6/ttk
\end{verbatim}
\endgroup
%% \end{Verbatim}

\begingroup \fontsize{10pt}{10pt}
\selectfont
%%\begin{Verbatim} [commandchars=\\\{\}]
\begin{verbatim} 
lappend ${::auto_path} {/usr/share/doc/libsqlite3-tcl}

# (iv (setq tmp (wg-sexa2dec "+40:05:26" )))   40.090437158469946

# (iv (setq tmp (wg-sexa2dec "105:12:15" )))   105.20409836065573

# use tcl associative array called hack: with (members):
set ::hack(FOCALLEN) 988 ; # Explore Scientific FOCAL LENGTH
set ::hack(OBSGEO-B) 105.20409
set ::hack(OBSGEO-L) 40.09044
set ::hack(OBSGEO-H) 1662.0
set ::hack()

proc wsplit {str sepStr} {
    if {![regexp $sepStr $str]} {
        return $str}
    set strList {}
    set pattern (.*?)$sepStr
    while {[regexp $pattern $str match left]} {
        if { "$left" != "" } {
          lappend strList $left
        }
        regsub $pattern $str {} str
    }
    if { "$str" != "" } {
      lappend strList $str
    }
    return $strList
}; # set foo [wsplit "+40:30   20.5" {[: ]+}]

proc d2s {arg} {
  set parts [wsplit $arg {[: hmsd]}]
  while {[string len $parts] < 3} {
    lappend parts {0}
  }
  set ans [expr {[lindex $parts 0] + ([lindex $parts 1] / 60.0) * ([lindex $parts 1] / 3600.0)}]
  return $ans
}; # d2s "+40:30:20.5"


proc r2s {arg} {
  set parts [wsplit $arg {[: hmsd]}]
  while {[string len $parts] < 3} {
    lappend parts {0}
  }
  set ans [expr { 15.0 * ([lindex $parts 0] + ([lindex $parts 1] / 60.0) * ([lindex $parts 1] / 3600.0))}]
  return $ans
}; # r2s "+17:30:20.5"


\end{verbatim}
\endgroup
%% \end{Verbatim}


%
