\section{IRAF Overview} \label{sec:IRAFOverview}

\subsection*{Quick Notes for IRAF 2.18}

The program may be installed in various ways. For Ubuntu 22.04, the 
apt library version is the old 2.17 code. It is easy to download and
build the source code. See section{sec:InstallFeatherNest} for details.

IRAF now installs in a different way. There is a program called \dhl{mkiraf}
that will create a \$HOME/.iraf directory. 

There are principle files: \dhl{/etc/iraf/login.cl} and \dhl{\$HOME/.iraf/loginuser.cl}
that come into play. The first, login.cl, should never be edited. If the file
\$HOME/.iraf/loginuser.cl exists it will be loaded. It is \$HOME/.iraf/loginuser.cl
where a user configures the startup environment's packages and user specific
tasks etc to be used. It is not uncommon to have a few of these files, say
one for photometry and the other for spectroscopy. A few variations that
handle different observatory's data. 

Details of these files will be dealt with later in Section \ref{sec:LoginScriptsInFull}.

IRAF \dhl{(Image Reduction and Analysis Framework)} is a collection of
'packages', each with a handful of 'tasks'. Each task has its own
parameter lists (.par and param files). IRAF is from the age of the
command-line-interface, and uses the approach identical to an modern
bash shell. The initial 'command-language' (cl), was extended to ecl
based on experience and needs. The desire to add functionality
resulted in PyRAF -- the adoption of Python 2.x environment. The
Python readline function was changed; syntax that matched that of ecl
was adapted to more than 95 percent of needs. All of the features of
Python were then available in the same way as using Python
interactivly.  This makes PyRAF the environment to use. Knowledge and
use of 'cl' directly is still needed.

Python is NOT a programming language, it is an ``environment''. With key
specifications included in the moving parts, it is impossible to distribute
a ``Python'' program -- one has the build the Python project into or
alongside any other environments on their platform. \index{Python!not program.}

There are a great many variances between individual instruments. A GUI
can only encapsulate settings with little ability to make the
necessary changes required between datasets to meet changing
conditions during a night.\dhl{[ Managing these differences are covered in
Section \ref{sec:IRAFGUINotes}}.

IRAF/PyRAF is based on physics from the 1980's and
90's \cite{1973asqu.book.....A}. Use of the Analysis facilities require
careful attention or should be avoided in lieu of modern algorithms
such as those that are emerging with Astropy. The analysis facilities
are reasonable for small-telescope, low resolution spectroscopy.

The IRAF/PyRAF interface uses the dated XWindows X11/R6 system. Most
of the desired GUI like features are provided by the X11 interface.
\index{X11}

The SAOImage/DS9 (ds9) \cite{1990BAAS...22..935V}
\cite{2019zndo...2530958J} program replaces XImtool, providing the
external task for image viewing and task interaction management. Ds9
uses both the XPA and SAMP protocols to facilitate interprocess
communications between the IRAF tasks, PyRAF interfaces, and the
images. SAMP provides tie-ins to Aladin and TOPCAT. This suite of
tools greatly assists in areas of photometry.  Python supports XPA
directly.

Once an instrument's configuration is well in hand, many of the interactive
steps may be skipped, expediting data reduction. A ``cl'' or PyRAF script
may be used to perform these tasks.

IRAF's main scripting started with the ``command line'' or \dhl{cl} script
engine. This was replaced with ``extended command line'' or \dhl{ecl}.
The desire for more power with scripting lead to the development
of PyRAF \cite{2006hstc.conf..437G,PyrafProgrammersGuide}. 

Under CL it is possible to write packages and tasks. A project is underway
to write cmosproc -- a basic reduction step sequence for cmos camera's.
The amount of things to learn with menu systems. pulldowns, and text
entry boxes is about as complex as simply typing a number of commands
into a script.

IRAF works with several image formats, principally the Flexible Image
Transport System or ``FITS'' A concise history may be found in \cite{FITSBirthday}.


IRAF/PyRAF3 may be run naively under Microsoft Windows, using the
professional or better versions of Win10/Win11. The professional OS
versions are needed for Microsoft's Hyper-V environment. For example
installing Ubuntu 22.04 in the HyperV as a virtual machine and using XMing
to handle the rendering of X11 windows.

IRAF makes heavy use of X11 and Unix commands. It is not well suited
for use with Windows. Currently IRAF/PyRAF3 runs quite well on a
Raspberry Pi 4 B2. It runs well on Apple Mac/OS by installing XQuartz
on the Macbooks. Success may be achieved using XMing or other
X-Windows Server programs on Windows. It is possible to host the full
IRAF/PyRAF system on a Raspberry Pi or other remote Linux machine and
perform reductions and analysis on a Mac or Windows machine.

From another Linux system simply create a terminal using \dhl{ssh -X
  myuser@pi.local}.  Programs like \dhl{XQuartz} for Macs; or
\dhl{Xming} and \dhl{PuTTY} for Win10 allow one to use the power of
X11 via windows acting like an X-Terminal! IRAF/PyRAF3 has been ported
to Apple OS both Intel and the new Mx processors as well as most
robust mainstream Linux OSs.

\section{Configuration loginuser and pyraflogin}

There are some tasks and python scripts (2.7) that are used to make
scripting easy.
\begin{quote}
{\color{red}
\begingroup \fontsize{8pt}{8pt}
\selectfont
%%\begin{Verbatim} [commandchars=\\\{\}]
\begin{verbatim} 
task $ds9="$foreign"
task $iraffind="$foreign"
task $r = ("$(ls -lt * | grep -v '[ ][.]' | head -10 ;)")  # show 10 recent changed files
task $ll = ("$(ls -1 $* | tee l.l; cat l.l)")
task $fvl  = ("$(~/bin/fitsverify -l $* | less -i)")
\end{verbatim}
\endgroup
%% \end{Verbatim}
}
\end{quote}

These python functions are 'sourced' by importing the functions from
the \$HOME/iraf/pyraflogin3.py file. The bash alias for pyraf
sets the PYTHONPATH environment variable, or you can:

\dhl{sys.path.append(getenv('HOME')+'/iraf')}

Then simply:

\dhl{from pyraflogin3 import *}

and the functions are available.


the iraf/pythonlogin3.py script
by the alias for pyraf. Some are for writing a new file or cube;
plotting; doing some helper tasks.

\begin{quote}
{\color{red}
\begingroup \fontsize{10pt}{10pt}
\selectfont
%%\begin{Verbatim} [commandchars=\\\{\}]
\begin{verbatim} 
def newfits(filename,data,header=None)
def newcube(filename,filelist,header=None)
def shiftup(filename,count=1)
def shiftside(filename,count=1)

def our_figure()
def our_histogram(mydata,title='Histogram')
def our_simple_plot(aline,x=None,title='Linear Plot',degree=None)
def our_plot(aline,x=None,degree=None,legend=None)
def our_plot_show(title='Linear Plot',usegrid=True)
def our_scatter_plot(x,y,title=None,spkeywords={})
def our_plot3d(z)

def pypar(task)
def pyparload(task,filename)
def pyfixbinning(imagename, newx,newy,xkwbin,ykwbin
def r2s (pobjra)
def d2s (pobjdec)
def s2r(rastr)
def s2d(decstr)
def fixCCDSOFT(filename)
def fakedispersion(filename,ypos=466,width=10)
def pyhelp(func)
\end{verbatim}
\endgroup
%% \end{Verbatim}
}
\end{quote}


\subsection{pyraflogin} \label{sec:pyraflogin}

This version of IRAF comes by way of a \dhl{conda channel} and requires
a \dhl{source activate iraf27} command to adjust the launch-point's terminal's
environment. 

The snap command requires these two directories:

/home/wayne/anaconda3/bin \\
/home/wayne/anaconda3/envs/iraf27/bin: \\
/home/wayne/anaconda3/envs/iraf27/iraf/unix/bin.linux

\begingroup \fontsize{10pt}{10pt}
\selectfont
%%\begin{Verbatim} [commandchars=\\\{\}]
\begin{verbatim} 
os.environ["PATH"] = '/home/wayne/anaconda3/envs/iraf27/bin:/home/wayne/anaconda3/bin:\\
   /home/wayne/anaconda3/envs/iraf27/iraf/unix/bin.linux'+os.environ["PATH"]
\end{verbatim}
\endgroup
%% \end{Verbatim}

An alias is used to launch pyraf.

\begingroup \fontsize{10pt}{10pt}
\selectfont
%%\begin{Verbatim} [commandchars=\\\{\}]
\begin{verbatim} 
alias pyraf="(export PYTHONSTARTUP=$HOME/iraf/pyraflogin.py; export PATH=$PATH:$HOME/iraf/smtsci/bin:$PATH; $HOME/anaconda3/envs/astrocondairaf/bin/pyraf -s; )"
\end{verbatim}
\endgroup
%% \end{Verbatim}

