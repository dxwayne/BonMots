\section{Coordinates}

\subsection{Image}

Here an image is considered to be a raw or pre-reduced image from a sensor.
The senaor reports values (x,y) in terms of its chip readout direction. The
image may contain one of more of overscans of columns and/or rows and a
sub-area with the data. The image may be padded. \index{Coordinates!physical}
\index{Coordinates!logical}

The region of interest of the image, where the science data resides, may
span the entire image or may consist of a section. Common sections include
BIASSEC, TRIMSEC, DATASEC etc.  \index{Coordinates!BIASSEC} \index{Coordinates!TRIMSEC}
\index{Coordinates!DATASEC}.

\subsection{Sections}

A section consists of two ranges of values, one for X and one for Y. They
appear as ASCII text within the header. The section is enclosed by square
brackets (\dhl{[]}) with the X and Y range separated by a comma. The
start and end of the region are inclusive, are enumerated starting with
1 (FORTRAN Convention), and up to and including NAXISi. The range may include
a wildcard character (\dhl{*}) taken to mean all or at least the rest of
the value range. Given NAXIX1=1000 and NAXIS2=700, some examples of a
section are:

\begingroup \fontsize{10pt}{10pt}
\selectfont
%%\begin{Verbatim} [commandchars=\\\{\}]
\begin{verbatim} 
[1:800,300:500]  # say the trace area of a spectra.
[1:*,*]          # equilavent to [*,*]
\end{verbatim}
\endgroup
%% \end{Verbatim}

