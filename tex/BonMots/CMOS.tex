\section{CMOS Chips}

CMOS sensors have a few extra details of operation, the
effective-gain, a pedistal\index{keywords!PEDISTAL}, and need to
calculate the GAIN \index{keywords!GAIN}.

In general a EGAIN of 1 is desirable with these cameras. The dark
noise is a bit higher, but the dynamic range \index{CMOS!dynamic range}
is better. A larger pedistal is required ($\sim$ 3000). This needs
to be stated.

% Jim Grubb ASI294

From \url{https://astronomy-imaging-camera.com/faqhttps://astronomy-imaging-camera.com/faq} section on amp glow:

\begin{itemize}
\addtolength{\itemsep}{-0.5\baselineskip}
   \item   pixels are charged at start, photons deplete that charge
   \item   actual gain on pixel
   \item   A pixel is a very complex structure:
\begin{itemize}
\addtolength{\itemsep}{-0.5\baselineskip}
   \item   at least one, often many, ADC
   \item   clock generators and power supply regulators
\end{itemize}
   \item   on-die processing
   \item   secondary processor
   \item   DDR buffer
\end{itemize}

Common wisdom is to set the EGAIN value to 120
Add a pedistal of 3000 in one case.


Have to qualify the general values for GAIN and RDNOISE

Calibrated sensors (voltages not light)
Can not use last night's cal images.

Exercises related to pattern and PRNU (ha) noise.

Post software assumptions (darks, flats etc).

Flats - pixel variation
        Dust, optical and vignetting



 

\begin{quote}
When it comes to CMOS cameras, ``amp glow'' is usually not from an
amplifier. CMOS sensors are usually “fully integrated” which means
that, unlike a CCD, readout electronics are included on the sensor die
along with all the pixels themselves. Each sensor has at least one,
often many, ADC (analog to digital conversion) and CDS (noise
reduction) units on it. There are also other support circuits on the
sensor die itself these days ... clock generators and power supply
regulators and such. These support circuits can generate heat or may
even emit NIR light, both of which can cause glows. Additionally, many
modern CMOS sensors include high performance image processing as part
of the sensor package, either in the form of on-die processing or a
secondary processor that is directly integrated into the sensor by
attaching it (often to the reverse side of the sensor.) This
processing circuitry can often generate heat that may produce glows.

\hskip{2cm }\url{https://astronomy-imaging-camera.com/faq}
\end{quote}

\subsection{QHY 600M}
>51ke- Full Well at 3.76um, >80ke- at extend mode at 3.76um

-USB3.0
-2.5 FPS, full frame, 16-bit images
-4.0 FPS, full frame, 8-bit image
-Support for ROI at higher frame rates


\begin{table}[h!]
%\phantomsection
%\addcontentsline{toc}{section}{ TOC CAPTION}
% \setlength{\belowcaptionskip}{6pt} % adjust space under caption abovecaptionskip
% \renewcommand{\arraystretch}{1.3} % adjust line spacing
%\small{
%\begin{minipage}{\textwidth}     % for footnotes in table.
%\caption[TOC]{Well Capacites QHY 600M}
\centering
\begin{tabular}{| l | l | l | l |}
%\MakeShortVerb{\|}
%\multicolumn{n}{fmt}{text for merged cols}
\hline
Full Well Capacity      &  1x1, &   2x2,  &    3x3    \\ 
\hline
Standard Mode           &  >51ke-&  >204ke-&  >408ke-    \\ 
Extend Full Well Mode   &  >80ke- & >320ke-&  >720ke-    \\ 
%% ones-based: \cline{a-b}
\hline
%%\DeleteShortVerb{|}
\end{tabular}
%%\end{minipage}    %% for footnotes  r@{.}l 
\caption[TOC]{Well Capacites QHY 600M}
\label{table:WellCapacitesQHY600M}
%%} % end small etc
\end{table}

Telescope Interface: M54/0.75











%
