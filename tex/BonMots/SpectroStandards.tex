\section{Create a IRAF Standard Star}

The calibration files consist of lines with wavelengths in Angstroms,
calibration magnitudes, and bandpass widths in Angstroms. The
magnitudes are converted to absolute flux per unit frequency using the
parameter fnuzero defined by:

\begin{subequations}
\begin{align}
F_{\nu} &= F_{\nu_{0}} \times\; 10^{-0.4\; m}
\end{align}
\end{subequations}

where $F_{\nu}$ is the flux in $ergs/cm2/s/hz$; $F_{\nu_{0}}$; and $m$ is the table
reported magnitude. 

Thus, fnuzero is the flux at magnitude zero. The flux units are
determined by this number. The default value was chosen such that Vega
at 5556 Angstroms has a magnitude of 0.048 and a flux of 3.52e-20
ergs/cm2/s/hz. This is the same value that was used by all previous
versions of this task.

5500.00    7.51  50.

% (iv (setq tmp (* -0.4 7.51  )))    -3.004  
3.4292660E-9  ergs/cm2/s/A

% (ive (setq tmp (/ constant-c  5.554e-5  )))    5.397776e+14

\subsection{The synphot Directory}

StSCI released an Astropy version of Synphot. Lets see if that works
for us.

\dhl{conda install synphot -c conda-forge}

The values are in ``photlam'' or photons per $\lambda$. 

The energy is:

\begin{subequations}
\begin{align}
E &= h\nu \\
  &= \frac{c}{\lambda}
\end{align}
\end{subequations}

The table entry in stsdas/lib/synphot/vega\_98\_to\_04.dat

          5550.     3.4292660E-9

\url{https://astro.uni-bonn.de/~sysstw/lfa_html/iraf/noao.onedspec.standard.html}

\url{https://synphot.readthedocs.io/en/latest/index.html#a-brief-history}

\url{https://specutils.readthedocs.io/en/stable/api/specutils.analysis.template_logwl_resample.html#specutils.analysis.template_logwl_resample}

\subsection{Flux Calibration Files}

There are two types of flux calibration files used by the task
STANDARD.

Standard Star Calibration Files

This type of file is any file that does not contain the parameter
"type" with a value of "blackbody". The only other parameter used by
this type of calibration file is the "units" parameter for the
wavelength units. If the units are not specified then the wavelengths
default to Angstroms. All older calibration files will have no
parameter information so they are interpreted as standard star
calibration files with wavelengths in Angstroms.

The calibration files consist of lines with wavelengths, calibration
magnitudes, and bandpass widths. The magnitudes are $m_AB$ defined as

\begingroup \fontsize{10pt}{10pt}
\selectfont
%%\begin{Verbatim} [commandchars=\\\{\}]
\begin{verbatim} 
m_AB(star) = -2.5 * log10 (f_nu) - 48.60
\end{verbatim}
\endgroup
%% \end{Verbatim}

where $f_nu$ is in$ erg/cm^2/s/Hz$. The $m_AB$ calibration magnitudes are
converted to absolute flux per unit frequency using the parameter
fnuzero of the STANDARD task. This conversion is defined by

$Fnu = fnuzero * 10. ** (-0.4 * magnitude)$

Thus, fnuzero is the flux at $m_AB$ of zero. The flux units are
determined by this number. The default value was chosen such that Vega
at 5556 Angstroms has a magnitude of 0.048 and a flux of 3.52e-20
ergs/cm2/s/Hz.

Blackbody Calibration Files

This type of file has the comment parameter "type" with a value of
"blackbody". It must also include the "band" and "weff" comment
parameters. If no "units" comment parameter is given then the default
units are Angstroms.

The rest of the file consists of lines with wavelengths, $m_AB$ of a
zero magnitude star (in that band magnitude system), and the bandpass
widths. The $m_AB$ are defined as described previously. Normally all the
$m_AB$ values will be the same though it is possible to adjust them to
produce a departure from a pure blackbody flux distribution.

The actual $m_AB$ calibration magnitudes for the star are obtained by the relation

$m_AB(star) = mag + m_AB(m=0) - 2.5 * log10 (B(weff,teff)/B(w,teff))$

where m is the magnitude of the star in the calibration band,
$m_AB(m=0)$ is the calibration value in the calibration file
representing the magnitude of a m=0 star (basically the $m_AB$ of Vega),
weff is the effective wavelength for the calibration file, and teff is
the effective temperature of the star. The function B(w,T) is the
blackbody function in $f_nu$ that provides the shape of the
calibration. Note how the normalization is such that at weff the last
term is zero and $m_AB(star) = m + m_AB(m=0)$.

The $m_AB(star)$ computed using the calibration values and the blackbody
function are then in the same units and form as for the standard star
files. The conversion to $f_nu$ and the remaining processing proceeds in
the same way as for standard star calibration data.

The parameters Imag and teff are specified by the user when running the STANDARD task.

Note that it is important that the flux values are in magnitudes. If
the fluxes are not given in magnitudes, as occurs with some files
imported from other sources, the STANDARD task will behave
incorrectly.


