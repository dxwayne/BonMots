% BonMots

\section{Install/Feather the Nest} \label{sec:InstallFeatherNest}

If you have previously installed IRAF, rename (mv) your directory to
the side \dhl{mv \$HOME/iraf \$HOME/mmDDyyyy-iraf }.

To install the IRAF 2.18 NOIRLab edition \index{Install IRAF} \index{IRAF!install}
you really must install the package off your home directory.

\begingroup \fontsize{10pt}{10pt}
\selectfont
%%\begin{Verbatim} [commandchars=\\\{\}]
\begin{verbatim} 
cd $HOME
git clone https://github.com/iraf-community/iraf.git
cd iraf
./install
\end{verbatim}
\endgroup
%% \end{Verbatim}

There are \dhl{TWO} directories to pay attention to:
\dhl{\$HOME/.iraf} and \dhl{\$HOME/iraf}.

Install the \dhl{pyraf3} package:

\begingroup \fontsize{10pt}{10pt}
\selectfont
%%\begin{Verbatim} [commandchars=\\\{\}]
\begin{verbatim} 
cd $HOME/.iraf
python3 -m venv ./venv; 
cd venv;
source bin/activate;
echo "made a new env at $(pwd) and activated it.";
pip3 install pyraf
\end{verbatim}
\endgroup
%% \end{Verbatim}


Then, \dhl{cd \$HOME/iraf} and use github to clone our helpers right into
the  \dhl{cd \$HOME/iraf} directory.

Install the provided login.cl into \dhl{\$HOME/.iraf} (note: it's ``dot''iraf).

ltodo{iraf login}{Brew up a flexmkiraf script to handle these chores.}

There are a number of helper files that need to be edited and copied
into various places within the filesystem. This is best done by hand
so you learn about the files and may maintain precise control over
their content.

Review and modify the \dhl{.cl} files as you copy them.

- The \dhl{bin} directory:

  -- \dhl{iraffind} - locate iraf files using the Unix find command the -iname
     switch and a full / partial part of the file's name. Pipe to \dhl{grep -v}
     as needed to winnow spuriuos results.

  -- \dhl{hlp2pdf} - convert help files into pdfs where possible. IRAF help pages
     use their own markup language. This script is a last-resort where one maynot
     find the one of about 1350 \dhl{.hlp} files.

\subsection{Site specific files}

A few site specific files should be changed, or modified. One is the
\dhl{obsdb.dat} file with details about the observatory satisfying the
\dhl{iraf.observarory} format and the \dhl{observat[ory] mysite} command.
Another is a file describing cameras and filters used by the site. These
are thorny.\index{key files!obsdb.dat}



\subsection{Files Summary}

/usr/lib/iraf/noao/imred/ccdred/ccddb/kpno/camera.dat - map internal IMAGETYPs
to iraf vocabulary. Surprising! \index{key files!camera.dat}

ccddb/kpno has exmample files for use with KPNO instruments. The
content and format is informative.


