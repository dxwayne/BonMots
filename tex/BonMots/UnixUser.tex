\section{Unix User Tidbits}

\ltodo{Networking}{Beef this up a lot!}

This section discusses basic Unix use commands and some exoteric aspects
of the Unix file system.

\subsection{Environment Variables}

Both bash and IRAF/PyRAF use environment variables, though the syntax varies
between bash and IRAF. \index{Unix!environment variables}

set
export

IRAF environment variables follow a convention of the text first and
the dollar sign at the end.

show

\index{IRAF/PyRAF!environment variables}


\subsection{Basic Commands} \lable{sec:BasicUnixCommands}

ls
cat
less
ps
grep
sed
awk
tr


\subsection{Basic Shell Programming} \label{sec:BasicShellProgramming}

tesing
   if [ ... ] and [[ ... ]]
do done
for ... in; 
printf
exit
case ...






\subsection{Networking}  \label{sec:UnixUserNetworking}

In Ye Olden Days, the \dhl{ifconfig} command was the go to tool to
query information and perform some control over networks. This
was replaced by the \dhl{ip} command as complexity of networking
grew.

The new \dhl{ip} command vs the old \dhl{ifconfig} is listed below.
The \dhl{ss} \index{network!ss sockets} command investigates the mind
numbing details about sockets.

The programs: arp, ifconfig, netstat and route are in the \dhl{apt install net-tools}
package.

\begin{table}[h!]
\centering
\begingroup \fontsize{10pt}{10pt}
\selectfont
\begin{tabular}{| l | l |}
\hline
ifconfig command & New ip command   \\
\hline
arp -a                              & ip neigh    \\ 
arp -v                              & ip -s neigh    \\ 
arp -s 192.168.1.1 1:2:3:4:5:6      & ip neigh add 192.168.1.1 lladdr 1:2:3:4:5:6 dev eth1    \\ 
arp -i eth1 -d 192.168.1.1          & ip neigh del 192.168.1.1 dev eth1    \\ 
ifconfig -a                         & ip addr    \\ 
ifconfig eth0 down                  & ip link set eth0 down    \\ 
ifconfig eth0 up                    & ip link set eth0 up    \\ 
ifconfig eth0 192.168.1.1           & ip addr add 192.168.1.1/24 dev eth0    \\ 
ifconfig eth0 netmask 255.255.255.0 & ip addr add 192.168.1.1/24 dev eth0    \\ 
ifconfig eth0 mtu 9000 i            & ip link set eth0 mtu 9000    \\ 
ifconfig eth0:0 192.168.1.2         & ip addr add 192.168.1.2/24 dev eth0    \\ 
netstat                             & ss    \\ 
netstat -neopa                      & ss -neopa    \\ 
netstat -g                          & ip maddr    \\ 
route                               & ip route    \\ 
\hline
\end{tabular}
\endgroup
\caption[Net-tools vs IProute2]{A ifconfig vs ip command quick summary.}
\label{table:ifconfigvsipcommand}
\end{table}
