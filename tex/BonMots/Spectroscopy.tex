%\subsection
\section{Spectroscopy}

The installation of the spectrograph requires focusing the instrumend
and adjusting the grating to cause the desired spectral range to fall 
onto the sensor.

The critical step is to focus of the collimator to guarantee parallel
angle's of incidence. The next step is to assure the science camera
is focused at infinity, as each wavelength of the dispersion beam
is parallel is collimated in its own unique way. The dispersion is
a continuous function that is discretely sampled by each pixel. The
pixel is its own sensor, with a starting wavelength and a small delta
wavelength. 

Spectroscopic reduction builds on the standard reduction approach by
subtracting suitable zero and dark images. For CMOS, use a
surface-smoothed (imsurfit) \index{IRAF!imsufit} image. For newer
CCDs and CMOS cameras, dark frames add noise.

The file types are:

\begin{table}[h!]
%\phantomsection
%\addcontentsline{toc}{section}{ TOC CAPTION}
% \setlength{\belowcaptionskip}{6pt} % adjust space under caption abovecaptionskip
% \renewcommand{\arraystretch}{1.3} % adjust line spacing
%\small{
%\begin{minipage}{.8\textwidth}     % for footnotes in table.
%\caption[TOC]{Image Types}
\centering
\begin{tabular}{| l | l |}
%\MakeShortVerb{\|}
%\multicolumn{n}{fmt}{text for merged cols}
\hline
IMAGETYP  & Purpose   \\
\hline
focus             & setup    \\ 
zero              & remove image noise    \\ 
dark              & remove image noise    \\ 
calibration (cal) & provide wavelength calibration    \\ 
reference         & provide flux calibration    \\ 
flat              & instrument response, remove fringing, longslit work    \\ 
%% ones-based: \cline{a-b}
\hline
%%\DeleteShortVerb{|}
\end{tabular}
%%\end{minipage}    %% for footnotes  r@{.}l 
\caption[TOC]{Image Types}
\label{table:ImageTypes}
%%} % end small etc
\end{table}

\newpage
In general, a matched set of images is required for each science image
reduction

Basic Steps:

\begingroup \fontsize{10pt}{10pt}
\selectfont
%%\begin{Verbatim} [commandchars=\\\{\}]
\begin{verbatim} 
ll t_*Bias*fits > /dev.null
!rm -f zeromaster.fits s_zeromaster.fits  # remove or additional extension added (bad)
zerocombine @l.l output=zeromaster.fits combine=median scale=mode
imsurfit input=zeromaster.fits output=s_zeromaster.fits 
      xorder=1 yorder=3 border=10 cross_terms=yes
imean = float(iraf.imstat(images='@l.l',fields='mean',format=iraf.no,Stdout=1)[0])
\end{verbatim}
\endgroup
%% \end{Verbatim}

\subsection{Reducing the Spectra}

It is recommended that for each new target, a matched set of comparison
and flat images be taken. This keeps the distortions of temperature
expansion, flexure, and optical axis shifts in line -- all dependent
on the orentation of the telescope and the resulting gravity vector.
If a rotator is used, then the rotator will shift the impact of the
gravity vector.

The flat is added to the system using an internal flat. It does no
good to slew to the location of the dome flat screen as the telescope
orientation will differ from that of the target.

For simplicity the file names match their purpose.

The main ideas:
\vspace{-.15cm}
\begin{enumerate}\addtolength{\itemsep}{-0.5\baselineskip}
%\setcounter{enumi}{N}
   \item   flat.fits take the flat exposure
   \item   comp.fits take the comparison exposure
   \item   sci.fits  the science exposure
\end{enumerate}

The goal is to extract the trace and add its data to the database for
the file. (The file names do need to be different for more than one
spectra -- again the use of flat/comp/sci is to keep the discussion
clear). Then using the dispersion trace from the science, extract the
pixels corresponding to the science trace from the flat, and also from
the comp. We differ from most IRAF discussions here by extracting an
additional spectrum from a flat.

\begin{figure}
\begingroup \fontsize{10pt}{10pt}
\selectfont
%%\begin{Verbatim} [commandchars=\\\{\}]
{\color{darkred}
\begin{verbatim} 
    sys.path.append(os.getenv('HOME') + "/iraf") # our helper path
    import pyraflogin3                           # import our helpers
\end{verbatim}
}
\endgroup
%% \end{Verbatim}
\caption{Use our pyraflogin3 script -- stored alongside login.cl and
loginuser.cl.}
\label{fig:pyraflogin3}
\end{figure}


\begin{figure}
\begingroup \fontsize{10pt}{10pt}
\selectfont
%%\begin{Verbatim} [commandchars=\\\{\}]
{\color{darkred}
\begin{verbatim} 
    apall     sci.fits output = sci.0001.fits (sci.ms.fits) # get disp
    apall     comp.fits ref=sci interactive=no # comp.0001.fits
    apall     flat.fits ref=sci interactive=no # flat.0001.fits 
    identify  comp.0001.fits coordlist=...      # per usual
    hedit     sci.0001.fits REFSPEC1=comp.0001.fits add+ update+ ver- show-
    dispcor   sci.0001.fits cal_sci.0001.fits # add the WCS to the file.
\end{verbatim}
}
\endgroup
%% \end{Verbatim}
\caption{The basic extraction steps for a star's observation.}
\label{fig:SpecRedSteps}
\end{figure}


Next move to the reference star's set of files and reduce per the
sequence in figure \ref{fig:SpecRedSteps}. Here we presume the slit
position is different -- hence the need for the comp and flat files to
be taken at this new location with new circumstances of temperature
and flexure.

So now we have two flat values. We may now choose to iterate,
wavelength-bin by wavelenght-bin, and adjust the science target's
spectrum to the reference star's spectrum. This makes for a flux to
flux basis -- just like correcting for vignetting in photometry.

\begin{figure}
\begingroup \fontsize{10pt}{10pt}
\selectfont
%%\begin{Verbatim} [commandchars=\\\{\}]
{\color{darkred}
\begin{verbatim} 
with fits.open('refflat.0001.fits'), fits.open('sciflat.0001.fits') as sci,ref:
   refdata = ref[0].data
   scidata = sci[0].data
   corrsci = sci.data / (refdata/scidata)
   newfits('corrsci.0001.fits',corrsci, scidata[0],header)
\end{verbatim}
} % color
\endgroup
%% \end{Verbatim}
\caption{Use python to flat the science target to the reference target.}
\label{fig:FlatSciRef}
\end{figure}

There are some 'hidden' commands:

\begingroup \fontsize{10pt}{10pt}
\selectfont
%%\begin{Verbatim} [commandchars=\\\{\}]
\begin{verbatim} 
    :/title Object and dateobs
    :/subtitle Instrument
    :/comment Observer/etc
    :/sysid no
\end{verbatim}
\endgroup
%% \end{Verbatim}


General workflow:
\vspace{-.15cm}
\begin{enumerate}\addtolength{\itemsep}{-0.5\baselineskip}
   \item   Make directories
   \item   Move RawData into the workspace
   \item   Copy in any necessary files into usw/
\end{enumerate}


DISPAXIS

trim images


apall

- fitting:

\begingroup \fontsize{10pt}{10pt}
\selectfont
%%\begin{Verbatim} [commandchars=\\\{\}]
\begin{verbatim} 
#g <newline> one of H,J,K,L to redefine newline
#then choose two of these separated by a comma
    d  Ratio (y / f)
    f  Fitted values
    r  Residuals of fit (y - f)
    n  Nonlinear part of data (linear component of fit subtracted)
    x  Indepedent variable
    y  Dependent variable (data being fit)
\end{verbatim}
\endgroup
%% \end{Verbatim}


\subsection{Steps}

Fix the gain of the image, with ASI296 camera drop the padded
values, and use a gain of 66000. This brings the science back
by about 2\%.

Fix the headers:


\begingroup \fontsize{10pt}{10pt}
\selectfont
%%\begin{Verbatim} [commandchars=\\\{\}]
\begin{verbatim} 
    hedit *fits DISPAXIS 1 add+ update+ ver- show-
    hedit *fits GAIN <right value>  add+ update+ ver- show-
\end{verbatim}
\endgroup
%% \end{Verbatim}


Trim image of excess X low values where QE is way too low, using the
python script. This will write a logical/physical WCS into the header.

\subsection{Focus}

\begingroup \fontsize{10pt}{10pt}
\selectfont
%%\begin{Verbatim} [commandchars=\\\{\}]
\begin{verbatim} 
!ds9 -tile mode row a10001_AD_Leo_East_Calibration_300s_20220212_052528m1.fits       a10002_AD_Leo_East_Calibration_300s_20220212_080927m1.fits a10003_AD_Leo_West_Calibration_300s_20220212_090214m1.fits &


#start ds9 
imexamine  
# j on a comp line (graphics window up)
# :epar  # save and quit
## naverage 11
## center,bacground - yes
## width 10   wings of the fit.
## xorder 0  median fit the background
## rplot 31  (big pixels gets by with less)
## point mode: show each sample as point, or draw a curve (szmarker 1 for small pts)



Package: TV, task jimexam

autoscale   h           Adjust number of histogram bins to avoid aliasing
background  jkr.        Subtract background for radial plot and photometry?
center      jkr.        Find center for radial plot and photometry?
eparam      cehjklrsuv. Edit parameters
naverage    cjkluv      Number of columns, lines, vectors to average
radius      r.          Radius of object aperture for radial plot and photmetry
round       cehjklruv.  Round axes to nice values?
rplot       jkr.        Radius to plot in 1D and radial profile plots
sigma       jk          Initial sigma for 1D gaussian fits
width       jkr.        Width of background region
x [min max] chjklruv.   Range of x to be plotted (no values for autoscaling)
y [min max] chjklruv.   Range of y to be plotted (no values for autoscaling)

banner      cehjklrsuv. Include standard banner on plots?
box         cehjklruv.  Draw box around graph?


logx        chjklruv.   Plot x axis logrithmically?
logy        chjklruv.   Plot y axis logrithmically?
majrx       cehjklruv.  Number of major tick marks on x axis
majry       cehjklruv.  Number of major tick marks on y axis
marker      chjklruv.   Marker type for graph
minrx       cehjklruv.  Number of minor tick marks on x axis
minry       cehjklruv.  Number of minor tick marks on y axis
pointmode   chjkluv     Plot points instead of lines?
szmarker    chjklruv.   Size of marks for point mode
ticklabels  cehjklruv.  Label ticks?
title       cehjklrsuv. Optional title for graph
unlearn     cehjklrsuv. Unlearn parameters to default values
xlabel      cehjklrsuv. Optional label for x axis
xorder      jkr.        X order of surface for background subtraction
ylabel      cehjklrsuv. Optional label for y axis
\end{verbatim}
\endgroup
%% \end{Verbatim}



%%\subsection{Splot Help}

Handy tasks: 

\begin{table}[h!]
%\phantomsection
%\addcontentsline{toc}{section}{ TOC CAPTION}
% \setlength{\belowcaptionskip}{6pt} % adjust space under caption abovecaptionskip
% \renewcommand{\arraystretch}{1.3} % adjust line spacing
%\small{
%\begin{minipage}{\textwidth}     % for footnotes in table.
%\caption[TOC]{Splot Key Sequences}
\centering
\begin{tabular}{| l | l |l |}
%\MakeShortVerb{\|}
%\multicolumn{n}{fmt}{text for merged cols}
\hline
Action  & Key Sequence & Effect  \\
\hline
   fit continuum:  & \llbox{t-fq}                           & continuum fit        \\ 
   write NEW file  & \llbox{i} \dhl{cont\_<newfilename>}    & 'write' new contunuum file \\
\hline
   window:         & cursor at left edge\llbox{a} cursor at right edge \llbox{a} & zoom window  \\ 
\hline
   title:          & \dhl{:/sysid no} \crreturn             & ~ ~ remove existing  \\ 
                   & \dhl{:/title <txt>} \crreturn          & ~ ~ line 1 title     \\ 
                   & \dhl{:/subtitle <subtxt>} \crreturn    & ~ ~ line 2  subtitle \\ 
                   & \dhl{:/comment <commenttxt>} \crreturn & ~ ~ line 3 comment   \\ 
                   & \llbox{r}                              & SHOW NEW TITLE       \\ 
\hline
  see values:      & \dhl{spacebar} & show x,values... \\
%% ones-based: \cline{a-b}
\hline
%%\DeleteShortVerb{|}
\end{tabular}
%%\end{minipage}    %% for footnotes  r@{.}l 
\caption{Splot Key Sequences}
\label{table:SplotKeySequences}
%%} % end small etc
\end{table}

\begingroup \fontsize{10pt}{10pt}
\selectfont
%%\begin{Verbatim} [commandchars=\\\{\}]
\begin{verbatim} 
? - This display                          r - Redraw the current window
/ - Cycle thru short help on stat line    s - Smooth (boxcar)
a - Autoexpand between cursors            t - Fit continuum(*)
b - Toggle base plot level to 0.0         u - Adjust coordinate scale(*)
c - Clear and redraw full spectrum        v - Velocity scale (toggle)
d - Deblend lines using profile models    w - Window the graph
e - Equiv. width, integ flux, center      x - Connects 2 cursor positions
f - Arithmetic functions: log, sqrt...    y - Plot std star flux from calib file
g - Get new image and plot                z - Expand x range by factor of 2
h - Equivalent widths(*)                  ) - Go to next spectrum in image
i - Write current image as new image      ( - Go to previous spectrum in image
j - Fudge a point to Y-cursor value       # - Select new line/aperture
k - Profile fit to single line(*)         % - Select new band
l - Convert to F-lambda                   $ - Toggle wavelength/pixel scale
m - Mean, RMS, snr in marked region       - - Subtract deblended fit
n - Convert to F-nu                       , - Down slide spectrum
o - Toggle overplot of following plot     . - Up slide spectrum
p - Convert to wavelength scale           I - Interrupt task immediately
q - Quit and exit                   <space> - Cursor position and flux

(*) For 'h' key: Measure equivalent widths
    a - Left side for width at 1/2 flux   l - Left side for continuum = 1
    b - Right side for width at 1/2 flux  r - Right side for continuum = 1
    c - Both sides for width at 1/2 flux  k - Both sides for continuum = 1

(*) For 'k' key: Second key may be used to select profile type
    g - Gaussian, l - Lorentzian, v - Voigt, all others - Gaussiank

(*) For 't' key: Fit the continuum with ICFIT and apply to spectrum
    / = normalize by the continuum fit
    - = subtract the continuum fit (residuals)
    f = replace spectrum by the continuum fit
    c = clean spectrum of rejected points
    n = do the fitting but leave the spectrum unchanged
    q = quit without fitting or modifying spectrum

(*) For 'u' key: Adjust the coordinate scale by marking features
    d = apply doppler correction to bring marked feature to specified coord.
    l = set linear (in wavelength) coordinates based on two marked features
    z = apply zero point shift to bring marked feature to
	specified coordinate

The colon commands do not allow abbreviations.

:# <comment>      - Add comment to log file
:dispaxis <val>   - Change summing parameter for 2D images
:log	          - Enable logging to save_file
:nolog            - Disable logging to save_file 
:nsum <val>       - Change summing parameter for 2D images
:show	          - Show full output of deblending and equiv. width measurments
:units <value>	  - Change coordinate units (see below)

:label  <label> <format> - Add label at cursor position
:mabove <label> <format> - Add tick mark and label above spectrum
:mbelow <label> <format> - Add tick mark and label below spectrum
    The label must be quoted if it contains blanks.  A label beginning
    with % (i.e. %.2f) is treated as a format for the x cursor position.
    The optional format is a gtext string (see help on "cursors").
    The labels are not remembered between redraws.

:auto [yes|no]    - Enable/disable autodraw option
:zero [yes|no]    - Enable/disable zero baseline option
:xydraw [yes|no]  - Enable/disable xydraw option
:hist [yes|no]    - Enable/disable histogram line type option
:nosysid [yes|no] - Enable/disable system ID option
:wreset [yes|no]  - Enable/disable window reset for new spectra option
:flip [yes|no]    - Enable/disable dispersion coordinate flip
:overplot [yes|no]- Enable/disable permanent overplot mode

:/help  Get help on GTOOLS options
:.help	Get help on cursor mode options


				UNITS

The units are specified by strings having a unit type from the list below
along with the possible preceding modifiers, "inverse", to select the
inverse of the unit and "log" to select logarithmic units. For example "log
angstroms" to plot the logarithm of wavelength in Angstroms and "inv
microns" to plot inverse microns.  The various identifiers may be
abbreviated as words but the syntax is not sophisticated enough to
recognized standard scientific abbreviations except as noted below.

	   angstroms - Wavelength in Angstroms
	  nanometers - Wavelength in nanometers
	millimicrons - Wavelength in millimicrons
	     microns - Wavelength in microns
	 millimeters - Wavelength in millimeters
	  centimeter - Wavelength in centimeters
	      meters - Wavelength in meters
	       hertz - Frequency in hertz (cycles per second)
	   kilohertz - Frequency in kilohertz
	   megahertz - Frequency in megahertz
	    gigahertz - Frequency in gigahertz
	         m/s - Velocity in meters per second
	        km/s - Velocity in kilometers per second
	          ev - Energy in electron volts
	         kev - Energy in kilo electron volts
	         mev - Energy in mega electron volts
		   z - Redshift

	          nm - Wavelength in nanometers
	          mm - Wavelength in millimeters
	          cm - Wavelength in centimeters
	           m - Wavelength in meters
	          Hz - Frequency in hertz (cycles per second)
	         KHz - Frequency in kilohertz
	         MHz - Frequency in megahertz
	         GHz - Frequency in gigahertz
		  wn - Wave number (inverse centimeters)

The velocity and redshift units require a trailing value and unit defining the
velocity zero point.  For example to plot velocity relative to
a wavelength of 1 micron the unit string would be:

	km/s 1 micron
\end{verbatim}
\endgroup
%% \end{Verbatim}

