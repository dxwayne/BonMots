\section{Identify}

Some import aspects for small telescope science relates
to the camera's pixel size. Measure and change here as needed.

\begingroup \fontsize{10pt}{10pt}
\selectfont
%%\begin{Verbatim} [commandchars=\\\{\}]
\begin{verbatim} 
        crval = ""              Approximate coordinate (at reference pixel)
        cdelt = ""              Approximate dispersion
        (nsum = "10")           Number of lines/columns/bands to sum in 2D images
       (match = 10.0)           Coordinate list matching limit
 (maxfeatures = 50)             Maximum number of features for automatic identification
      (zwidth = 100.0)          Zoom graph width in user units
       (ftype = "emission")     Feature type
      (fwidth = 9.0)            Feature width in pixels
     (cradius = 19.0)           Centering radius in pixels
   (threshold = 10.0)           Feature threshold for centering
      (minsep = 10.0)           Minimum pixel separation
\end{verbatim}
\endgroup
%% \end{Verbatim}

Usually a night will have the same approximage \dhl{crval} and
\dhl{crpix} values.  After the first solution, these may be entered
for second/subsequent solutions.



\begingroup \fontsize{10pt}{10pt}
\selectfont
%%\begin{Verbatim} [commandchars=\\\{\}]
\begin{verbatim} 
lpar identify
       images = "comp.0001.fits" Images containing features to be identified
        crval = ""              Approximate coordinate (at reference pixel)
        cdelt = ""              Approximate dispersion
     (section = "middle line")  Section to apply to two dimensional images
    (database = "database")     Database in which to record feature data
   (coordlist = "near.dat")     User coordinate list
       (units = "angstroms")    Coordinate units
        (nsum = "10")           Number of lines/columns/bands to sum in 2D images
       (match = 10.0)           Coordinate list matching limit
 (maxfeatures = 50)             Maximum number of features for automatic identification
      (zwidth = 100.0)          Zoom graph width in user units
       (ftype = "emission")     Feature type
      (fwidth = 9.0)            Feature width in pixels
     (cradius = 19.0)           Centering radius in pixels
   (threshold = 10.0)           Feature threshold for centering
      (minsep = 10.0)           Minimum pixel separation
    (function = "chebyshev")    Coordinate function
       (order = 5)              Order of coordinate function
      (sample = "*")            Coordinate sample regions
    (niterate = 1)              Rejection iterations
  (low_reject = 3.0)            Lower rejection sigma
 (high_reject = 3.0)            Upper rejection sigma
        (grow = 0.0)            Rejection growing radius
   (autowrite = no)             Automatically write to database
    (graphics = "stdgraph")     Graphics output device
      (cursor = "")             Graphics cursor input
     (aidpars = "")             Automatic identification algorithm parameters
        (mode = "al")           
\end{verbatim}
\endgroup
%% \end{Verbatim}
