\subsection{Apall Command Summary} \label{sec:ApallCommandSummary}

\begingroup \fontsize{10pt}{10pt}
\selectfont
%%\begin{Verbatim} [commandchars=\\\{\}]
\begin{verbatim} 
       		 APEXTRACT CURSOR KEY SUMMARY

?  Print help             j  Set beam number        u  Set upper limit(s)
a  Toggle all flag        l  Set lower limit(s)     w  Window graph
b  Set background(s)      m  Mark aperture          y  Y level limit(s)
c  Center aperture(s)     n  New uncentered ap.     z  Resize aperture(s)
d  Delete aperture(s)     o  Order ap. numbers      I  Interrupt
e  Extract spectra        q  Quit                   +  Next aperture
f  Find apertures         r  Redraw graph           -  Previous aperture
g  Recenter aperture(s)   s  Shift aperture(s)      .  Nearest aperture
i  Set aperture ID        t  Trace aperture(s)      

       		 APEXTRACT COLON COMMAND SUMMARY

:apertures      :center         :npeaks         :show           :t_width
:apidtable      :clean          :nsubaps        :skybox         :threshold
:avglimits      :database       :nsum           :t_function     :title
:b_function     :extras         :order          :t_grow         :ulimit
:b_grow         :gain           :parameters     :t_high_reject  :upper
:b_high_reject  :image          :peak           :t_low_reject   :usigma
:b_low_reject   :line           :plotfile       :t_naverage     :weights
:b_naverage     :llimit         :r_grow         :t_niterate     :width
:b_niterate     :logfile        :radius         :t_nlost        :write
:b_order        :lower          :read           :t_nsum         :ylevel
:b_sample       :lsigma         :readnoise      :t_order        
:background     :maxsep         :saturation     :t_sample       
:bkg            :minsep         :shift          :t_step         

		APEXTRACT CURSOR KEYS

?    Print help
a    Toggle the ALL flag
b an Set background fitting parameters
c an Center aperture(s)
d an Delete aperture(s)
e an Extract spectra (see APSUM)
f    Find apertures up to the requested number (see APFIND)
g an Recenter aperture(s) (see APRECENTER)
i  n Set aperture ID
j  n Set aperture beam number
l ac Set lower limit of current aperture at cursor position
m    Define and center a new aperture on the profile near the cursor
n    Define a new aperture centered at the cursor
o  n Enter desired aperture number for cursor selected aperture and remaining
     apertures are reordered using apidtable and maxsep parameters
     (see APFIND for ordering algorithm)
q    Quit
r    Redraw the graph
s an Shift the center(s) of the current aperture to the cursor position
t ac Trace aperture positions (see APTRACE)
u ac Set upper limit of current aperture at cursor position
w    Window the graph using the window cursor keys
y an Set aperture limits to intercept the data at the cursor y position
z an Resize aperture(s) (see APRESIZE)
.  n Select the aperture nearest the cursor to be the current aperture
+  c Select the next aperture (in ID) to be the current aperture
-  c Select the previous aperture (in ID) to be the current aperture
I    Interrupt task immediately.  Database information is not saved.

The letter a following the key indicates if all apertures are affected when
the ALL flag is set.  The letter c indicates that the key affects the
current aperture while the letter n indicates that the key affects the
aperture whose center is nearest the cursor.

			APEXTRACT COLON COMMANDS

:show [file]	   Print a list of the apertures (default file is STDOUT)
:parameters [file] Print current parameter values (default file is STDOUT)
:read [name]       Read apertures from database (default to the current image)
:write [name]      Write apertures to database (default to the current image)

The remaining colon commands are task parameters and print the current
value if no value is given or reset the current value to that specified.
Use :parameters to see current parameter values.
\end{verbatim}
\endgroup
%% \end{Verbatim}
