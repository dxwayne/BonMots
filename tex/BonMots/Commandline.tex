% /home/wayne/play/sasiraf/doc/BonMots/Commandline.tex

\section{IRAF/PyRAF Command Line Basics}

The login.py and loginuser.py file declares a lot of Unix functions
to be \dhl{foreign} \index{functions!foreign}, foreign as not residing
within the IRAF source code or affiliated packages. This includes any
programs you may want to write. 

Non-foreign commands are available by preceeding/escaping the
line with an exclaimation point, aka a ``bang'' (!). \index{functions!bang}.

Common commands that will be used within the CL/PyRAF environment.

\begin{table}[h!]
\centering
\begin{tabular}{| l | l | l |}
\hline
Command  & Description   & PyRAF Form   \\
\hline
pwd       & print wd            & !\$(pwd)    \\ 
ls        & list files          & !ls *pat > l.l    \\ 
cd        & change dir          & change     \\ 
cp        & copy                & ! cp -pr *pat dir    \\ 
mkdir     & mkdir               & !mkdir p     \\ 
!rm       &                     & remove (*)    \\ 
cat       & dump file           & pipes    \\ 
less/more & like page           & help task  | less    \\ 
awk       & strong inline edits & gawk    \\ 
sed       & script-ed           & pipe partner    \\ 
grep      & filters             & grep -v (not pattern)    \\ 
diff      & compare files       & compare source files    \\ 
sh        & use bash            & really, use bash    \\ 
%% ones-based: \cline{a-b}
\hline
%%\DeleteShortVerb{|}
\end{tabular}
%%\end{minipage}    %% for footnotes  r@{.}l 
\caption[Basic Linux Commands]{A handful of basic Linux commands.}
\label{table:BasicCommands}
%%} % end small etc
\end{table}


In loginuser.cl

iraf//noao/imred/ccdred/ccddb/
iraf/noao/lib/obsdb.dat

setinst.site='gao' -- defined in iraf/noao/lib/obsdb.dat

camera.dat - description tying loca; dialects to IRAF

\begingroup \fontsize{10pt}{10pt}
\selectfont
%%\begin{Verbatim} [commandchars=\\\{\}]
\begin{verbatim} 
exptime         otime
darktime        ttime
imagetyp        data-typ
subset          f1pos
biassec         biassec []
datasec         datasec []

fixpix          bp-flag 0
overscan        bt-flag 0
zerocor         bi-flag 0
darkcor         dk-flag 0
flatcor         ff-flag 0
fringcor        fr-flag 0

'OBJECT (0)'            object
'DARK (1)'              dark
'PROJECTOR FLAT (2)'    flat
'SKY FLAT (3)'          other
'COMPARISON LAMP (4)'   other
'BIAS (5)'              zero
'DOME FLAT (6)'         flat
\end{verbatim}
\endgroup
%% \end{Verbatim}

\subsection{Linux Commands}

Linux is an full modern multi-user peer-peer operating system that is very
similar in all regards to Unix.

Unix proper does not distinguish between binary or text files\footnote{mode
of 'r' has been replaced of late with the need to open 'rb', or read-binary.}.
File extensions are ``social conventions'' and should carry no meaning.
There may be multiple periods (dots) in a file name. However, while 
filenames may have spaces in Linux -- this places a burden on the
users to put silly-quote marks around the file names to make sure
all the space-separated parts stay together as one name.

All commands and command line arguments are case-sensitive. The filenames
Makefile and makefile are different -- indeed Makefiles usually control
making programs using sub-programs defined by their own makefile (note
case of the filename characters).


The paragigm of a Linux ``\dhl{program}'' or command is one of a
``\dhl{filter}''. A filter recognizes three files, that are 
open and available when the program stars : 

\vspace{-.15cm}
\begin{enumerate}\addtolength{\itemsep}{-0.5\baselineskip}
%\setcounter{enumi}{N}
   \item  standard in (stdin),
   \item  standard out (stdout) and 
   \item  standard error (stderr).
\end{enumerate}

A Filter takes its data-input from stdin, uses arguments to affect (or
not) the data-input read for that file and sends the results to stdout.
Errors are reported on a separate stream called stderr. This disambiguates
a few important messages from a possible torrent of output.

Techinally -- each program, when it wakes up, has an unsigned integer
count of the arguments and a pointer to an array of strings that are
the 'arguments'. The arguments may be used in any fashion by the program.
A technical description may be found in Section \ref{sec:technicaldescription}.

\subsection{Linux Command Technical Description}  \label{sec:technicaldescription}

This is a crude Bakus-Naur Format (BNF) parser template for a linux command
line processor, where:

\begin{quote}
   STRING is a set of [A-Za-z\_][A-Za-z\$\_0-9.-], essentially no spaces
   and being careful not to use characters that carry weight with more
   advanced features.

   DASH is the hyphen character (minus sign)

   DOUBLEDASH is two immediately adjacent hyphens.

   wildcard -- left to another discussion, but essentiall an asterisk or a 
   question mark

   special character [\$()\&|<>`'"] as \$ for environment variables, () to initiate
   a subshell, \&\& logical and || logical or >2\&1 to copy stderr into stdout,
   back-tick (`) as in `cmd` to take the results of cmd into the command line.

   The special characters like the arrows, page up/down etc are not considered
   and have been known to create filenames that are a Real Pain\texttrademark to remove!


\end{quote}

are some of the uses for characters w.r.t. command lines.

Here is the syntax: example follows...

\begingroup \fontsize{10pt}{10pt}
\selectfont
%%\begin{Verbatim} [commandchars=\\\{\}]
\begin{verbatim} 
Program      :: command arguments
             ;

command      :: STRING
             ;

arguments    :: argument
             | argument SPACE argument
             | wildcard
             ;

options      :: option
             | filename
             ;

option       :: shortoption
             | longoption
             ;

shortoption  :: DASH singlechar             # eg -H -i -n -s
             | shortoption | singlechar     # eg -Hins
             ;

longoption   :: concat(DOUBLEDASH,switchname)                  # eg --with-filename (-H)
             | concat(DASH,switchname)
             | longoption switchoption
             ;

switchname   :: STRING
             ;

switchoption :: STRING
             ;

switchoption :: 

filename     :: STRING
             |  QUOTEDSTRING
             ;

\end{verbatim}
\endgroup
%% \end{Verbatim}

\textbf{\emph{Goal:}} Find the filename and line numbers for procedure definitions in
all the tcl files at or under this directory; while we're at it
keep a copy of that list and then view the list.

\textbf{\emph{Example:}}
\verb=find $(PWD) -name "*.tcl" | xargs grep -Hins "^[ ]*proc" | tee results.txt | less=

\textbf{\emph{Prose:}}. Find all files (\verb=-iname "*.tcl"=) at or
below \verb=$(PWD)= sending the filenames one at a time into a
Linux program \verb=xargs= that will feed the files -- one at a time -- into
grep. Use \verb=grep= with witches that outputs the filename:lineno: text
to its output. But! Using plumbing terms, a \verb=tee= will send flow
in two directions: 1) on to the next task, and 2) into a file we
will call \verb=results.txt=. Then use \verb=less= to comfortably
view and search the results. 

Essentially, linux commands can be combined to leverage their varying
and various actions with parameters to quickly hash up a command sequence
that acts like a program. All right from the command line.

It can be made more difficult, but sufficent unto the day, etc.

Find a GUI that does that!


\subsection{IRAF/PyRAF Environment}  \label{sec:IRAFPyRAFEnvironment}

IRAF/PyRAF tasks refer to elements in the file system using a set of
cl specific environment variables. They are of the form of <text><dollarsign>.
They are set with the \dhl{set} command and may be examined with the \dhl{show}
command. They usually do not expand.

The command:

\dhl{x=iraf.show('iraf', Stdout=1)[0]}

will set a python variable with the text of the base directory for iraf.

In spectroscopy, the identify task wants to know the fully-qualified
path to the linelists file via the iraf.identify.coordlist variable or
the \dhl{coordlist} task parameter. This is a bit tedious to determine.
Using lpar identify, the coordlist = 'linelists\$fhydrogen.dat', so
having the proper value for linelists\$ is needed.

show linelist returns:  \dhl{noaolib\$linelists/}. Now the question of
what is the value of \dhl{noaolib\$}?. Using show noaolib returns
\dhl{noao\$lib/}, show noao returns \dhl{iraf\$noao} and show iraf
returns \dhl{/home/wayne/anaconda3/envs/geminiconda/iraf/}.

Putting it all together:

\dhl{/home/myhome/anaconda3/envs/geminiconda/iraf/noao/lib/linelists}

is the place to put your linelists, together with all the default IRAF
supplied linelists.


\textbf{\emph{Summary:}} Using \dhl{set} and \dhl{show} manages IRAF
environment variables. \index{IRAF!environment variables}.
