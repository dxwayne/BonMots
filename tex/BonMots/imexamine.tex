\subsection{Imexamine Details}

Imexamine can be used to test focus, smile, tilt and overall flatness.
\index{spectrum!imexam!focus} \index{spectrum!imexam!smile} \index{spectrum!imexam!tilt} 
The key lies is using the one-d fits (j and k keys). The parameters for these
fits are held in a series of sub-lpar files \dhl{jimexam} and \dhl{limexam}.
These control the number of pixels, width of things etc.

In general, once \dhl{imexamine} is started, you need to click inside the
\dhl{ds9} window and do one ``throw away'' j-click. Then inside the
\dhl{ds9} window use a \llbox{:} character AND then find the input
area within the IRAF graphics window. Then enter ``\dhl{epar jimexam} \crreturn'',
fill in the epar fields, and hit the save button. 


\begin{table}[h!]
\centering
\begin{tabular}{| l | l |l |}
\hline
Action  & Key Sequence & Effect  \\
\hline
focus  &  \llbox{j} <l-click new position> \llbox{o}\llbox{j} & same line, different frames \\
tilt   &  \llbox{j} move <l-click>  \llbox{o} \llbox{j}   & pick bright one first \\
smile  &  \llbox{j} move <l-click>  \llbox{o} \llbox{j} ... move <l-click>  \llbox{o} \llbox{j} & along the line \\
\hline
\end{tabular}
\caption{Imexam focus,tilt,smile tricks. The logfile records the x,y and fwhm.}
\label{table:Imexamfocustiltsmiletricks}
\end{table}


This is the output of the ``?'' command.

\begingroup \fontsize{10pt}{10pt}
\selectfont
%%\begin{Verbatim} [commandchars=\\\{\}]
\begin{verbatim} 
			    -- IMEXAMINE COMMANDS --

			   CURSOR KEY COMMAND SUMMARY

? Help              h Histogram         p Previous frame    x Coordinates
a Aperture Sum      i Image cursor      q Quit              y Set origin
b Box coords        j Line gauss fit    r Radial plot       z Print grid
c Column plot       k Col gauss fit     s Surface plot      , Quick phot
d Load display      l Line plot         t Output image      . Quick prof fit
e Contour plot      m Statistics        u Vector plot       
f Redraw            n Next frame        v Vector plot       
g Graphics cursor   o Overplot          w Toggle logfile    


			     COLON COMMAND SUMMARY

allframes    ceiling      iterations   naverage     pointmode    width
angh         center       label        nbins        radius       x
angv         constant     logfile      ncolumns     round        xformat
autoredraw   dashpat      logx         ncontours    rplot        xlabel
autoscale    defkey       logy         ncoutput     select       xorder
background   eparam       magzero      ncstat       szmarker     y
banner       fill         majrx        nhi          ticklabel    yformat
beta         fitplot      majry        nlines       title        ylabel
boundary     fittype      marker       nloutput     top_closed   yorder
box          floor        minrx        nlstat       unlearn      z1,z2
buffer       interval     minry        output       wcs          zero


                           OUTPUT OF 'a' AND 'r' KEYS

The 'a' key and logfile output has column labels and each object has one
line of measurements in the logfile and two lines on the terminal.  The 'r'
key shows only the second line on the status line and the information from
the first line is in the graph title.  The first line contains the x and y
center coordinates and optional world coordinates.  The second line
contains the aperture magnitude and flux, the estimated background sky, the
profile fit peak, the ellipticity and position angle from the moment
analysis, and four estimates of the profile width.  The four estimates are
from the moment analysis, the full-width enclosing half the flux, the
profile fit, and a direct estimate of the full width at half-maximum.


			      CURSOR KEY COMMANDS

?	Print help
a	Aperture radial photometry measurement (see above for output)
b	Box coordinates for two cursor positions - c1 c2 l1 l2
c	Column plot
d	Load the image display
e	Contour plot
f	Redraw the last graph
g	Graphics cursor
h	Histogram plot
i	Image cursor
j	Fit 1D gaussian to image lines
k	Fit 1D gaussian to image columns
l	Line plot
m	Statistics
	    image[section] npixels mean median stddev min max
n	Next frame or image
o	Overplot
p	Previous frame or image
q	Quit
r	Radial profile plot (see above for output)
s	Surface plot
t	Output image centered on cursor (parameters output, ncoutput, nloutput)
u	Centered vector plot from two cursor positions
v	Vector plot between two cursor positions
w	Toggle write to logfile
x	Print coordinates
	    col line pixval [xorign yorigin dx dy r theta]
y	Set origin for relative positions
z	Print grid of pixel values - 10 x 10 grid
,	Quick profile photometry measurement (Gaussian or Moffat)
.	Quick radial profile plot and fit (Gaussian or Moffat)

				COLON COMMANDS

Explicit image coordinates may be entered using the colon command syntax:

	:column line key

where column and line are the image coordinates and the key is one
of the cursor keys.  A special syntax for line or column plots is also
available as :c# or :l# where # is a column or line and no space is
allowed.

Other colon commands set or show parameters governing the plots and other
features of the task.  Each graph type has it's own set of parameters.
When a parameter applies to more than one graph the current graph is assumed.
If the current graph is not applicable then a warning is given.  The
"eparam" and "unlearn" commands may be used to change many parameters and
without an argument the current graph parameters are modified while with
the graph key as an argument the appropriate parameter set is modified.
In the list below the graph key(s) to which a parameter applies are shown.

allframes               Cycle through all display frames to display images
angh        s           Horizontal angle for surface plot
angv        s           Vertical angle for surface plot
autoredraw  cehlrsuv.   Automatically redraw graph after colon command?
autoscale   h           Adjust number of histogram bins to avoid aliasing
axes        s           Draw axes in surface plot?
background  jkr.        Subtract background for radial plot and photometry?
banner      cehjklrsuv. Include standard banner on plots?
beta        ar		Moffat beta parameter (INDEF to fit or value to fix)
boundary    uv          Boundary extension type for vector plots
box         cehjklruv.  Draw box around graph?
buffer      r.          Buffer distance for background subtraction
ceiling     es          Data ceiling for contour and surface plots
center      jkr.        Find center for radial plot and photometry?
constant    uv          Constant value for boundry extension in vector plots
dashpat     e           Dash pattern for contour plot
eparam      cehjklrsuv. Edit parameters
fill        e           Fill viewport vs enforce unity aspect ratio?
fitplot     r           Overplot profile fit on data?
fittype     ar          Profile fitting type (gaussian|moffat)
floor       es          Data floor for contour and surface plots
interval    e           Contour interval (0 for default)
iterations  ar          Iterations on fitting radius
label       e           Draw axis labels for contour plot?
logfile                 Log file name
logx        chjklruv.   Plot x axis logrithmically?
logy        chjklruv.   Plot y axis logrithmically?
magzero     r.          Magnitude zero for photometry
majrx       cehjklruv.  Number of major tick marks on x axis
majry       cehjklruv.  Number of major tick marks on y axis
marker      chjklruv.   Marker type for graph
minrx       cehjklruv.  Number of minor tick marks on x axis
minry       cehjklruv.  Number of minor tick marks on y axis
naverage    cjkluv      Number of columns, lines, vectors to average
nbins       h           Number of histogram bins
ncolumns    ehs         Number of columns in contour, histogram, or surface plot
ncontours   e           Number of contours (0 for default)
ncoutput                Number of columns in output image
ncstat                  Number of columns in statistics box
nhi         e           hi/low marking option for contours
nlines      ehs         Number of lines in contour, histogram, or surface plot
nloutput                Number of lines in output image
nlstat                  Number of lines in statistics box
output			Output image root name
pointmode   chjkluv     Plot points instead of lines?
radius      r.          Radius of object aperture for radial plot and photmetry
round       cehjklruv.  Round axes to nice values?
rplot       jkr.        Radius to plot in 1D and radial profile plots
select                  Select image or display frame
sigma       jk          Initial sigma for 1D gaussian fits
szmarker    chjklruv.   Size of marks for point mode
ticklabels  cehjklruv.  Label ticks?
title       cehjklrsuv. Optional title for graph
top_closed  h           Close last bin of histogram
unlearn     cehjklrsuv. Unlearn parameters to default values
wcs                     World coordinate system for axis labels and readback
width       jkr.        Width of background region
x [min max] chjklruv.   Range of x to be plotted (no values for autoscaling)
xformat			Coordinate format for column world coordinates
xlabel      cehjklrsuv. Optional label for x axis
xorder      jkr.        X order of surface for background subtraction
y [min max] chjklruv.   Range of y to be plotted (no values for autoscaling)
yformat			Coordinate format for line world coordinates
ylabel      cehjklrsuv. Optional label for y axis
yorder      r.          Y order of surface for background subtraction
z1          h           Lower intensity value limit of histogram
z2          h           Upper intensity value limit of histogram
zero        e           Zero level for contour plot
\end{verbatim}
\endgroup
%% \end{Verbatim}


\subsection{Example Output}

Here are some snapshots from runs....
\ltodo{imexam output}{expand the prose for these segments.}

\begingroup \fontsize{10pt}{10pt}
\selectfont
%%\begin{Verbatim} [commandchars=\\\{\}]
\begin{verbatim} 
#   COL    LINE    COORDINATES      R    MAG    FLUX     SKY    PEAK    E   PA BETA ENCLOSED   MOFFAT DIRECT
4214.77  474.98 4214.77 474.98  49.22  10.23 806432.   3712.   1283. 2.48   89 2.12    34.48    11.47   9.65

# [1] a10001_AD_Leo_East_Calibration_300s_20220212_052528m1.fits - 
Lines 460-470: center=4214.52 peak= 1133.6 sigma=5.285 fwhm=12.44 bkg=3629.09
# [2] a10002_AD_Leo_East_Calibration_300s_20220212_080927m1.fits - 
Lines 458-468: center=4209.73 peak=1199.61 sigma=4.951 fwhm=11.66 bkg=3645.09
# [3] a10003_AD_Leo_West_Calibration_300s_20220212_090214m1.fits - 
Lines 459-469: center=4205.55 peak=1064.48 sigma=5.219 fwhm=12.29 bkg=3613.09

%          lisp math                  shift wrt #2
% (iv (setq tmp (- 4209.73 4214.52)))    -4.790
% (iv (setq tmp (- 4209.73 4205.55)))     4.179

\end{verbatim}
\endgroup
%% \end{Verbatim}

\clearpage
\subsection{jimexam}
\begingroup \fontsize{10pt}{10pt}
\selectfont
%%\begin{Verbatim} [commandchars=\\\{\}]
\begin{verbatim} 
lpar jimexam
      (banner = yes)            Standard banner
       (title = "")             Title
      (xlabel = "wcslabel")     X-axis label
      (ylabel = "Pixel Value")  Y-axis label
    (naverage = 3)              Number of liness or columns to average
      (center = yes)            Solve for center?
  (background = yes)            Solve for background?
       (sigma = 1.0)            Initial sigma (pixels)
       (width = 9.0)            Background width (pixels)
      (xorder = 0)              Background terms to fit (0=median)
       (rplot = 21.0)           Plotting radius (pixels)
          (x1 = INDEF)          X-axis window limit
          (x2 = INDEF)          X-axis window limit
          (y1 = INDEF)          Y-axis window limit
          (y2 = INDEF)          Y-axis window limit
   (pointmode = yes)            plot points instead of lines?
      (marker = "plus")         point marker character?
    (szmarker = 1.0)            marker size
        (logx = no)             log scale x-axis
        (logy = no)             log scale y-axis
         (box = yes)            draw box around periphery of window
  (ticklabels = yes)            label tick marks
       (majrx = 5)              number of major divisions along x grid
       (minrx = 5)              number of minor divisions along x grid
       (majry = 5)              number of major divisions along y grid
       (minry = 5)              number of minor divisions along y grid
       (round = no)             round axes to nice values?
        (mode = "al")           
\end{verbatim}
\endgroup
%% \end{Verbatim}
\clearpage
\subsection{limexam}
\begingroup \fontsize{10pt}{10pt}
\selectfont
%%\begin{Verbatim} [commandchars=\\\{\}]
\begin{verbatim} 
lpar limexam
      (banner = yes)            Standard banner
       (title = "")             Title
      (xlabel = "wcslabel")     X-axis label
      (ylabel = "Pixel Value")  Y-axis label
    (naverage = 1)              Number of lines to average
          (x1 = INDEF)          X-axis window limit
          (x2 = INDEF)          X-axis window limit
          (y1 = INDEF)          Y-axis window limit
          (y2 = INDEF)          Y-axis window limit
   (pointmode = no)             plot points instead of lines?
      (marker = "plus")         point marker character?
    (szmarker = 1.0)            marker size
        (logx = no)             log scale x-axis
        (logy = no)             log scale y-axis
         (box = yes)            draw box around periphery of window
  (ticklabels = yes)            label tick marks
       (majrx = 5)              number of major divisions along x grid
       (minrx = 5)              number of minor divisions along x grid
       (majry = 5)              number of major divisions along y grid
       (minry = 5)              number of minor divisions along y grid
       (round = no)             round axes to nice values?
        (mode = "al")           
\end{verbatim}
\endgroup
%% \end{Verbatim}

