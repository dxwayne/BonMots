\section{Python}

Origninal PyRAF was written in python 2.7. The new PyRAF3 now uses python
up to 3.7.

\subsection{Changes}

The most important changes include shifting many python functions to return
an iterator rather that a list. This includes the \dhl{map} keyword.

\begingroup \fontsize{10pt}{10pt}
\selectfont
%%\begin{Verbatim} [commandchars=\\\{\}]
\begin{verbatim} 
def plusone (x) : return x+1;
map(plusone,[1,2,3,4]) 
\end{verbatim}
\endgroup
%% \end{Verbatim}

used to return [2,3,4,5], it now returns an iterator for a list.  This
is fixed with \dhl{list(map(plusone,[1,2,3,4]))}, but requires
visiting all usages in the code.  The move expedites \dhl{for loops}
and \dhl{in} statements, as ``most of the time'' the map is used with
an iterator.


The \dhl{print} keyword. Print is weird enough, its main deficency is formating
output. This is never easy, but python has never done a decent job compared
to the original ease of printf in "C". The form of the print statement is now
one of a function and requires its paremeters to be set off in parenthesis and
the output file be handled with a keyword argument.

print >>sys.stderr,"hello" becomes print("hello",file=sys.stderr). 

The \dhl{f-string} was introduced in python 3.6. Here, print parses
the output string and recognizes fields delimited by "{}" curly-braces.
The content of the curley braces accepts a program's variable name or
constant. Formating is achieved with a suffix starting with a colon
and followed by a C-like printf string. Details are in Python's PEP 498.

\ltodo{2.7x to 3.x Conversions}{Flesh out the {2.7x to 3.x Conversions}
