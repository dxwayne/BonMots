\section{FITS Files} \label{sec:FITSFiles}

Consult the FITS standard (HaHa), as FITS stands for ``\dhl{F}lexible
\dhl{I}mage \dhl{T}ransport \dhl{S}ystem''.  OK, people, its
\dhl{FLEXIBLE}.

Use Fitsverify \cite{fitsverify-1} before publishing.

Files uses ASCII character. Sorry Fran\c{c}ios. No \dhl{UNICODE}!

Headers have ``n'' card blocks, each block has 36 cards. Blank cards
are filled with 0x20. The old FITS spec permitted NULL (0x00)
characters, \dhl{BUT THIS IS NOT PERMITTED NOW}.

The last card in the 'deck', is in a block, and starts with END and
has 80-3=77 0x20 characters.  The rest of the cards, to bring the
block count to 36 are 80 times 0x20 characters.

At the end of that is where data starts.

Data is stored Big Endian until the end of the extent. The count of byte
is determined by the product of NAXISn values times the BSIZE keyword.

Get used to it. It's the way it works.

There are no guard values, etc. Think magnetic tape for storage of card decks.
A 2000 card deck, one box of cards, is heavy. Sufferable but still heavy.

Header cards are 80 columns precisely -- think old computer punch card. \footnote{You have
not lived until you use a card punch for hours.}

The card consists of a KEYWORD of up to 8 characters. The keyword starts in column
1 and is padded with 0x20 (ASCII space) chacacters.



There are several types of FITS files.

\begin{table}[h!]
%\phantomsection
%\addcontentsline{toc}{section}{ TOC CAPTION}
% \setlength{\belowcaptionskip}{6pt} % adjust space under caption abovecaptionskip
% \renewcommand{\arraystretch}{1.3} % adjust line spacing
%\small{
%\begin{minipage}{.8\textwidth}     % for footnotes in table.
%\caption[TOC]{Very Basic Definition for FITS Files}
\centering
\begin{tabular}{| l | l |}
%\MakeShortVerb{\|}
%\multicolumn{n}{fmt}{text for merged cols}
\hline
Type  & Desciption   \\
\hline
IMAGE                     & 1D NAXIS = 1, and NAXIS1 has data    \\ 
                          & Image 2D NAXIS=2 and NAXIS1=rows, NAXIS2=columns    \\ 
                          & Cube 3D+ Image; NAXIS=3, NAXISn=rows,columns,time    \\ 
TABLE                     & Contains tabular data with a set of columns, and    \\ 
                          & specified number of rows.    \\ 
Multi-Extenson-Fits (MEF) & A case where extensions are concatenated together to    \\ 
                          & form one file. Special header keywords apply.    \\ 
%% ones-based: \cline{a-b}
\hline
%%\DeleteShortVerb{|}
\end{tabular}
%%\end{minipage}    %% for footnotes  r@{.}l 
\caption[FITS Types]{Very Basic Definition for FITS Files}
\label{table:VeryBasicDefinitionforFITSFiles}
%%} % end small etc
\end{table}

The files consists of extents. One extent for a basic image file.
The format is a header section followed by a data section.



\ltodo{Add BIB/WEB info}{Add the references for the files.}

Some things to keep in mind:

\vspace{-.15cm}
\begin{enumerate}\addtolength{\itemsep}{-0.5\baselineskip}
%\setcounter{enumi}{N}
   \item  Keep file names short, it really helps. 
   \item  Load up the header, not the filename.
   \item  Avoid non-ASCII characters, and things like minus signs, more than one period,
etc. 
\end{enumerate}


There are plenty of sources for detailed information about fits files.

In this context, the important thing to remember is FITS files can contain
``EXTENTS''. IRAF may create an EXTENT during certain operations.

The task \dhl{zerocombine} will add the recent effort to the end of the
zeromaster.fits file -- causing that file to create errors with non-obvious
error messages later on in the process. Here you see the file(s) are
removed before the operation.

\begingroup \fontsize{10pt}{10pt}
\selectfont
%%\begin{Verbatim} [commandchars=\\\{\}]
\begin{verbatim} 
!rm rawzeromaster.fits zeromaster.fits
zerocombine @l.l output=rawzeromaster.fits combine=average scale=none reject=minmax
\end{verbatim}
\endgroup
%% \end{Verbatim}

\subsection{Fitsverify}

The fitsverify task, available from HEASARC, can be made to be a \dhl{foreign task}
and test fits files for validity.

