\section{Spectral SNR} \label{sec:Spectral SNR}
\ltodo{SNR}{Beef this up a lot}
\begin{quote}


Hello,In order to compute the SNR for a pixel in a 1D spectrum you
need an estimate of the uncertainty in the measured flux. 

This can be hard, particularly if the spectrum has been dispersion and
flux calibrated. There are two typical ways to estimate this.

One is for data taken on a 2D detector with known noise
characteristics. For example CCDs characterized by a readout noise and
a gain. This is a challenge because most 2D calibration operations
(dark subtraction, flat fielding, etc) modify the noise
characteristics and the uncertainties are not propagated by the
software (such as CCDPROC). So what is done in this case is estimate
an uncertainty using a noise model during the extraction of a 1D
spectrum from the 2D format. This is what APEXTRACT does in the "sigma
band". This is a spectrum recorded with the extracted spectrum based
on assuming a simple readout noise and gain with Poisson statistics is
correct for the each 2D pixel and then propagating this during the 2D
extraction. In summary, the uncertainty is estimated from the 2D
statistics and all pixels that contribute to the 1D spectrum value.The
second method is to estimate the uncertainty from the statistics of
the 1D pixels around the region of each pixel. 

The problem here is that this only makes sense with continuum regions
since lines should not contribute to the standard deviation of
pixels. In this case, which is conceptually easier and more
independent of modeling and previous calibration operations, usually
the best you can do is a typical noise value for all pixels and the
SNR is then the ratio of the measured spectrum divided by this noise
value at all wavelengths.

The latter is what is done with SPLOT and the 'm' key. The former
depends on using APEXTRACT as briefly mentioned in previous
postings. This discussion only gives you ideas for how to possibly
determine the SNR. To make a SNR spectrum there are a number of ways,
though these require several steps and possible some scripting to
automate. The tasks that might prove useful are CONTINUUM SARITH or
IMEXPR MIMSTAT.

The goal here is to determine either a single value or spectrum for
the standard deviation with lines and any continuum slopes
removed. Once your have that you use SARITH to compute the SNR
spectrum.I realize this is a recipe but hopefully you can devise an
image processing recipe or script appropriate for your data and
needs.

Yours, Frank Valdes 

An error band is an associated spectrum produced by the APEXTRACT
package tasks when extracting a 1D spectrum from a 2D format. It is
obtained when the parameter "extras" is "yes" and variance "weighting"
or "cleaning" is used. The spectrum is stored in a "multispec" format
which is a 3D image where the lines in the first "band" [*,*,1] are
the 1D spectra and the associated data is in the higher dimensional
bands. One of these would be the sigma spectra.

Frank Valdes \url{https://iraf.net/forum/viewtopic.php?showtopic=136062}
\end{quote}
