Hey,

You can separate the system into three parts: the telescope;
optics/optical paths within the spectrograph; the camera assy
bolted to the outside of the spectrograph. Remember that
what the sensor wants is a perfect, rectangular image
of the slit -- just in separated wavelenghts. 

A few things happen at the slit due to the telescope:

- the colvolved point-spread-function of the beam/atm/science

- angle of the converging beam (effective f/ratio) into slit

- the size of the 'star' at the slit, or sky angular
size for longslit work with nebula/galaxies

- astigmatisms including refractor's chromatic abberation

- vignetting

- effective surface area (flux and exposure time)

- smoothness/polish of grating's edges

- grating is metallic deposition on a substrate, quality
and optical properties of the substrate and plane-parallel
shifts etc.


Inside the spectrograph:

- geometry, matching the rectangular beam to the sensor.
Irregularities in the grating, dirt.

- astigmatism of the colliminating and camera ('objective')
lenses

- wavefront/wavelength properties of internal plane mirrors

- thermal expansion (linear) and thermal effects (torque
on the box messing with internal angles.)

The science sensor:

- the pixel size

- QE of the pixel

- manufacturing/perormance of each pixel

- dark current management, and anti-blooming aspects
of the sensor

- repeatability of the pixel over same conditions

- everyone's favorite cosmic rays



If the f/ratio of the incomming beam is 'smaller' it
will pass through a smaller colliminating lens but
only use a part of the grating. Consider a round
lens and a square grating. This leaves the corners
of the grating under illuminated. Conversely, if
the f/ratio is 'larger' parts may miss the colliminating
lens (results in loss of flux, longer exposure time,
like having a smaller aperture, but without a loss
of resolution in longslit work); the light cone is
bigger than the grating and loss of light (longer
exposure time) results.  The phrase the f/ratios
have to be the same is economics not science per se.

Downsides of the larger colliminating lens includes the dimeter
(larger housing), the diameter (longer focal length and longer
housing), the weight (impacts tracking), quality. Lenses suffer from
various abberations. Professional spectrographs use mirrors for
colliminating.

A few things: I use centimeters every where. The lines/mm
I use lines/cm. Wavelength of 5000A is 5e-5cm. 

Tabling the slit's width, look at the grating, how each
line of the grating is illuminated (round beam on square
grating means very short chords of the grating used
at the edges. This impacts the dispersion physics.)
The rough zero-th order resolution is simply R \equiv
W (width of grating meaning W * lines/cm).


